% -*-LaTeX-*-

% $Log: exercise-answers.tex,v $
% Revision 1.4  2007/12/25 18:00:39  stiber
% Changes and corrections for Winter 2008.
%
% Revision 1.3  2007/03/20 23:50:28  stiber
% Cleaned up for standalone text. Fixed various errors, updated LaTeX.
%
% Revision 1.3  2006/04/12 02:57:35  stiber
% Added an exercise to chapter 3.
% 
% Revision 1.2  2004/03/29 19:46:31  stiber
% Updated for Spring 2004 and new textbook (DSP First).
% 
% Revision 1.1  2004/02/19 00:20:01  stiber
% Initial revision
% 

\chapter{Answers to Self-Test Exercises}
\label{ap:exercise-answers}

\section{Chapter~\ref{ch:physical-signals}: Signals in the Physical World}
\label{sc:ch1ex}

\begin{enumerate}
\item Let's assume that we encode audio information as 8 bits
  per sample at 8000 samples per second (this would be considered very
  low quality). How many bits per second is that?\label{it:ch1ex1}

  \textit{Answer}: 64,000 bits/second.

\item Let's say we encode each frame in a
  video stream as 1000x1000 pixels, 24 bits/pixel, and 30 frames/second.
  How many bits per second is that?\label{it:ch1ex2}

  \textit{Answer}: 720 million bits/second.

\item We need to find a function of $x$ whose second derivative is proportional
  to itself. Can you think of one or two?\label{it:ch1ex3}

  \textit{Answer}: $\sin\omega t$ and $\cos\omega t$ both work.

\item Solve for $\omega$ in terms of $k$ and $m$.\label{it:ch1ex4}

  \textit{Answer}: $\omega = \sqrt{k/m}$.

\item Fork stiffness ($k$) clearly affects vibration frequency. As the
  fork get stiffer ($k$ increases), does the vibration frequency go up
  or down?\label{it:ch1ex5}

  \textit{Answer}: as $k$ increases, $\omega$ (which is vibration
  frequency) increases.

\item If the two forks were close
  together, we would heard the sum of their tones. What would that sum
  be?\label{it:ch1ex6}

  \textit{Answer}: $a_1 \cos(\omega t + \phi_1) + a_2
  \cos(\omega t + \phi_2)$, where $a_1$ and $a_2$ are the vibrational
  amplitudes and $\phi_1$ and $\phi_2$ are the two phases.

\item What does $\mathbf{u}+\mathbf{v}$ look like if
  $\mathbf{u}$ and $\mathbf{v}$ are rotating?\label{it:ch1ex7}

  \textit{Answer}: If
  the $\omega$s of $\mathbf{u}$ and $\mathbf{v}$ are equal, then
  $\mathbf{u}+\mathbf{v}$ rotates with them, the magnitudes staying
  constant. Therefore, $\mathbf{u}+\mathbf{v}$ is a sinusoid of the same
  frequency, $\omega$.

\item What is the sum of the two complex numbers $x + jy$ and $v +
  jw$?\label{it:ch1excn1}

  \textit{Answer}: You add the reals and imaginaries separately,
  yielding $(x+v) + j (y+w)$.

\item What is the product of the two complex numbers $x + jy$ and $v +
  jw$?\label{it:ch1excn2}

  \textit{Answer}: It works just like multiplying polynomials,
  $(x + jy)(v + jw) = xv + jyv + jwx + j^2yw = (xv-yw) + j(xw+yv)$
  (remember that $j^2 = -1$).

\item Convert the complex number $z = x + jy$ to polar form,
  $R\angle\theta$.\label{it:ch1excn3}

  \textit{Answer}: $R = |z| =  \sqrt{x^2+y^2}$; $\theta =
  \arctan(y/x)$.

\item Multiply the two polar-form complex numbers $R_1\angle\theta_1$ and
  $R_2\angle\theta_2$.\label{it:ch1excn4}

  \textit{Answer}: $R_1R_2\angle(\theta_1+\theta_2)$.

\item Multiply the two complex sinusoids $z_1$ and
  $z_2$.\label{it:ch1exce1}

  \textit{Answer}: Just like multiplying the polar
  representation of two vectors, $z_1z_2 =
  R_1R_2e^{j(\theta_1+\theta_2)}$.

\item The complex conjugate is indicated by $z^*$. If $z=x+jy$,
  $z^*=x-jy$.  What is the complex conjugate of the complex sinusoid,
  $z=Re^{j\theta}$?\label{it:ch1exce2}

  \textit{Answer}: Like the polar representation, it
  has the same magnitude but a negative angle, $z^* =
  Re^{-j\theta}$.

\item Answer the following for $z = x + jy$:\label{it:ch1exce3}
  \begin{enumerate}
  \item What is $z + z^*$?

    \textit{Answer}: $2\Real(z)$.
  \item What is $z - z^*$?

    \textit{Answer}: $2j\Imag(z)$.

  \item What is $zz^*$?

    \textit{Answer}: $|z|^2$.
  \end{enumerate}

\item Prove the relationship in~(\ref{eq:fs-cs3}).\label{it:ch1ex8}

  \textit{Answer}:
  \begin{equation*}
    c_{-k}^* = \left[\frac{1}{T}\int_0^T f(t)e^{jk\omega_0t}dt\right]^*
    = \frac{1}{T}\int_0^T f^*(t)e^{-jk\omega_0t}dt
    = \frac{1}{T}\int_0^T f(t)e^{-jk\omega_0t}dt
    = c_k
  \end{equation*}
  because $f(t) = f(t)^*$ ($f(t)$ is real).

\item From equation~(\ref{eq:fs-freal1}),
  derive~(\ref{eq:fs-freal2}).\label{it:ch1ex9}

  \textit{Answer}:
  \begin{align*}
    \Real [c_k e^{jk\omega_0t}]
    &= \Real[c_k\cos k\omega_0t + jc_k\sin k\omega_0t ]\\
    &= \Real(c_k)\cos k\omega_0t + \Real(jc_k)\sin k\omega_0t ]\\
    &= \Real(c_k)\cos k\omega_0t - \Imag(c_k)\sin k\omega_0t
  \end{align*}

\item Prove (\ref{eq:fs-S0}).\label{it:ch1ex10}

  \textit{Answer}: Since
  the denominator of $\sin\alpha/\alpha$ is zero at $\alpha=0$, we
  instead evaluate this as
  \begin{equation*}
    \sinc(0) = \lim_{\alpha\rightarrow 0}\sin\alpha/\alpha
  \end{equation*}
  and use L'H\^{o}pital's rule. The limit becomes
  \begin{align*}
    \sinc(0) & =  \left.\frac{\deriv{}{\alpha}\sin\alpha}
      {\deriv{\alpha}{\alpha}}\right|_{\alpha=0} \\
    & =  \left.\frac{\cos\alpha}{1}\right|_{\alpha=0} \\
    & =  1
  \end{align*}

\end{enumerate}

\section{Chapter~\ref{ch:computer-signals}: Signals in the Computer}
\label{sc:ch2ex}

\begin{enumerate}

\item The hubcap of a car coming to a stop in a motion
  picture.\label{it:ch2ex1}

  \textit{Answer}: Signal: the spokes; sampling: the discrete
  images in the movie.

\item A TV news anchor squirming while wearing a tweed jacket.\label{it:ch2ex2}

  \textit{Answer}: Signal: the tweed texture; sampling: the
  discrete frames of the video.


\item A helicopter blade while the helicopter is starting up on a
  sunny day.\label{it:ch2ex3}

  \textit{Answer}: Signal: the blade motion;
  sampling: the strobing effect as each blade blocks, then reveal, the
  sun.

\item In this case, $\mu=0$ and the variable's range is [-1/2, +1/2]
  LSB. The result is a standard deviation (equivalent to the RMS error
  computed in the textbook) of $\sigma = 1/\sqrt{12} \, \textrm{LSB}
  \approx 0.29 \, \textrm{LSB}$ (how?).\label{it:ch2ex4}

  \textit{Answer}: Given the value for $\mu$
  and the range of the (now definite) integral, we have:
  \begin{align*}
    \sigma^2 &= \int_{-1/2}^{+1/2} x^2 \mathrm{d}x \\
    &= \left. \frac{x^3}{3} \right|_{-1/2}^{+1/2} \\
    &= \frac{1}{24} + \frac{1}{24} \\
    &= \frac{1}{12}
  \end{align*}
  And so $\sigma=1/\sqrt{12}$.

\item If we use an 8-bit ADC, 5V corresponds
  to 255 and 1mV is then 0.051 LSB. (What is the SNR for the original
  signal?\label{it:ch2ex5}

  \textit{Answer}: It is $20 \log 5/0.001 = 74\textrm{dB}$,
  which is quite good.

\item What ratio of amplitudes is represented by one bel?\label{it:ch2ex6}

  \textit{Answer}: A bel is ten dB; and 10dB is an amplitude
  ratio of $\sqrt{10} \approx 3.2$.


\end{enumerate}

\section{Chapter~\ref{ch:filt-intro}: Filtering and Feedforward Filters}
\label{sc:ch3ex}

\begin{enumerate}

\item Is the signal of equation~\ref{eq:sine4} periodic? If so, what
  is its period?\label{it:ch3ex0}

  \textit{Answer:} Since this signal is a sum of sinusoids with
  frequencies that have a common divisor, then, yes, the signal is
  periodic. The greatest common divisor is 50Hz, and so the smallest
  period for the signal is $T=0.02$ seconds.

\item Since $|\cos(\omega\tau)|\leq 1$, the maximum value
  $|H(\omega)|$ can reach is $(1+a_1)$, which occurs when the angle
  $\omega\tau=n\pi, n=0,2,4,...$ (zero or even multiples of
  $\pi$). Why is this?\label{it:ch3ex1}

  \textit{Answser}: These are the values of $\omega\tau$ for which
  $\cos\omega\tau=+1$.

\item Use Euler's formula and the definition of the magnitude of a
  complex vector to derive~(\ref{eq:ff-mh})
  from~(\ref{eq:ff-mha}).\label{it:ch3ex2}

  \textit{Answer}: Subsituting $e^{-j\omega \tau} = \cos\omega\tau -
  j\sin\omega\tau$, we obtain $|H(\omega)| = |1+b_1 (\cos\omega\tau +
  j\sin\omega\tau)| = |1+b_1 \cos\omega\tau + j
  b_1\sin\omega\tau|$. The magnitude of a complex number is the square
  root of the sum of the squares of its real and imaginary parts, so
  $|H(\omega)| = |(1+b_1 \cos\omega\tau)^2 +
  b_1^2\sin^2\omega\tau|^\frac{1}{2}$. Computing the squared values,
  $|H(\omega)| = |1 + 2b_1\cos\omega\tau + b_1^2\cos^2\omega\tau +
  b_1^2\sin^2\omega\tau|^{\frac{1}{2}}$. Factor out the $b_1^2$ from
  the last two terms and remember that $\cos^2\theta + \sin^2\theta =
  1$, and you're home free.


\item Suppose that we sample a signal at 1000Hz. For each of the
  following analog frequencies $f$, determine $\omega$,
  $\hat{f}$, and $\hat{\omega}$. Indicate if that frequency will be
  aliased.\label{it:ch3ex2.2}
  \begin{enumerate}
  \item $f=100$Hz: $\omega = 2\pi f = 200\pi$ radians/sec.
    $\hat{f} = f/f_s = 100/1000 = 0.1$ cycles/sample.
    $\hat{\omega} = 2\pi \hat{f} = 0.2\pi$ radians/sample. Not aliased.
  \item $f=200$Hz: $\omega = 2\pi f = 400\pi$ radians/sec.
    $\hat{f} = f/f_s = 200/1000 = 0.2$ cycles/sample.
    $\hat{\omega} = 2\pi \hat{f} = 0.4\pi$ radians/sample. Not aliased.
  \item $f=500$Hz: $\omega = 2\pi f = 1000\pi$ radians/sec.
    $\hat{f} = f/f_s = 500/1000 = 0.5$ cycles/sample.
    $\hat{\omega} = 2\pi \hat{f} = \pi$ radians/sample. Not aliased.
  \item $f=1000$Hz: $\omega = 2\pi f = 2000\pi$ radians/sec.
    $\hat{f} = f/f_s = 1000/1000 = 1$ cycle/sample.
    $\hat{\omega} = 2\pi \hat{f} = 2\pi$ radians/sample. This is
    aliased, because it is greater than the Nyquist limit, $f_s/2 =
    500$Hz ($\hat{\omega}_{\mathit{Nyquist}} = \pi$).
  \end{enumerate}

\item Suppose that we sample a signal at 44.1kHz (the sampling rate
  used in audio CDs). For each of the following analog frequencies
  $f$, determine $\omega$, $\hat{f}$, and $\hat{\omega}$. Indicate if
  that frequency will be aliased.\label{it:ch3ex2.3}
  \begin{enumerate}
  \item $f=100$Hz: $\omega = 2\pi f = 200\pi$ radians/sec.
    $\hat{f} = f/f_s = 100/44100 \approx 0.0023$ cycles/sample.
    $\hat{\omega} = 2\pi \hat{f} \approx 0.014$ radians/sample. Not aliased.
  \item $f=1000$Hz: $\omega = 2\pi f = 2000\pi$ radians/sec.
    $\hat{f} = f/f_s = 1000/44100 \approx 0.023$ cycles/sample.
    $\hat{\omega} = 2\pi \hat{f} \approx 0.14$ radians/sample. Not aliased.
  \item $f=10000$Hz: $\omega = 2\pi f = 20000\pi$ radians/sample.
    $\hat{f} = f/f_s = 10000/44100 \approx 0.23$ cycles/sample.
    $\hat{\omega} = 2\pi \hat{f} \approx 1.4$ radians/sample. Not aliased.
  \item $f=20000$Hz: $\omega = 2\pi f = 40000\pi$ radians/sec.
    $\hat{f} = f/f_s = 20000/44100 \approx 0.45$ cycles/sample.
    $\hat{\omega} = 2\pi \hat{f} \approx 2.8$ radians/sample. Not aliased.
  \item $f=25000$Hz: $\omega = 2\pi f = 50000\pi$ radians/sec.
    $\hat{f} = f/f_s = 25000/44100 \approx 0.57$ cycles/sample.
    $\hat{\omega} = 2\pi \hat{f} \approx 3.6$ radians/sample. This is
    aliased, because it is greater than the Nyquist limit, $f_s/2 =
    22.05$kHz ($\hat{\omega}_{\mathit{Nyquist}} = \pi$).
  \end{enumerate}


\item Write equation~(\ref{eq:ff-manyk}) for $k=0, 1, 2, 3$, then
  write the transfer function for each.\label{it:ch3ex3}

  \textit{Answer}: For $k=0$, $y[n] = b_0 x[n]$ and $H(z) = b_0$;
  $k=1$, $y[n] = x[n] (b_0 + b_1 e^{-j\hat{\omega}})$ and $H(z) = b_0
  + b_1z^{-1}$; 
  $k=2$, $y[n] = x[n](b_0 + b_1 e^{-j\hat{\omega}} + b_2
  e^{-2j\hat{\omega}})$ and $H(z) = b_0 + b_1z^{-1} + b_2 z^{-2}$; 
  $k=3$, $y[n] = x[n](b_0 + b_1 e^{-j\hat{\omega}} + b_2
  e^{-2j\hat{\omega}} + b_3 e^{-3j\hat{\omega}})$ and $H(z) = b_0 +
  b_1z^{-1} + b_2 z^{-2} + b_3 z^{-3}$.

\item Given the signal $x(t) = \sin t$ and the derivative operator
  $D=\derivin{}{t}$, what is $Dx(t)$?\label{it:ch3ex4}

  \textit{Answer}: $Dx(t)=\cos t$.

\item When 
  \begin{align*}
    H_1(z) &= b_0 + b_1z^{-1}\\
    H_2(z) &= b'_0 + b'_1z^{-1}
  \end{align*}
  with $b_0$, $b_1$, $b'_0$, and $b'_1$ are constants, show that
  $H_2(z)H_1(z) = H_1(z)H_2(z)$.\label{it:ch3ex5}

  \textit{Answer}: Multiplying $H_1(z)H_2(z)$, we obtain
  $(b_0+b_1z^{-1})(b'_0+b'_1z^{-1})=b_0b'_0 + (b_0b'_1 +
  b_1b'_0)z^{-1} + b_1b'_1z^{-2}$. Multiplying $H_2(z)H_1(z)$, we get
  $(b'_0+b'_1z^{-1})(b_0+b_1z^{-1}) = b'_0b_0 + (b'_0b_1 +
  b'_1b_0)z^{-1} + b'_1b_1z^{-2}$.  Because multiplication is
  commutative, these two expressions are equal. In other words, series
  combination of filters is commutative because multiplication of
  polynomials is commutative.

\item Prove $|z^2|=1$ in
  equation~(\ref{eq:twostep-ex}).\label{it:ch3ex6}

  \textit{Answer}: $|z^2| = |e^{2j\omega}|$. From Euler's formula,
  this is $|\cos 2\omega + j\sin 2\omega|$. Since the magnitude of a
  complex number in rectangular form is the square root of the sum of
  the squares of its real and imaginary components, and $\cos^2
  2\omega + \sin^2 2\omega = 1$, $|z^2|=1$.

\item Starting with the factored magnitude response in
  equation~(\ref{eq:twostep-factored}), derive expressions for $b_1$
  and $b_2$ in terms of $z_1$ and $z_2$.\label{it:ch3ex7}

  \textit{Answer}: The factored magnitude response is
  $|H(z)|=|(z-z_1)(z-z_2)|$. Multiplying the two terms out yields
  $|H(z)|=|z^2 - zz_1 - zz_2 + z_1z_2|=|z^2 - (z_1+z_2)z +
  z_1z_2|$. Because $|H(z)|=|z^2 + b_1z + b_2|$,
  $b_1=-(z_1+z_2)$ and $b_2=z_1z_2$.

\item What abstract data type (ADT) should hold the delayed
  inputs?\label{it:ch3ex8}

  \textit{Answer}: a queue.

\end{enumerate}

\section{Chapter~\ref{ch:convolution}: The Z-Transform and Convolution}
\label{sc:ch6ex}

\begin{enumerate}

\item Determine the z-transform for the sequence
  $x[n]=\{1,2,5,7,0,1\}$, $n=0,1,2,3,4,5$\label{it:ch6ex1}

  \textit{Answer}:
  \begin{equation}
    X(z)=1+2z^{-1}+5z^{-2}+7z^{-3}+z^{-5}
  \end{equation}

\item Determine the z-transform of the sequence
  $x[n]=\{1,2,5,7,0,1\}$, $n=-2,-1,0,1,2,3$\label{it:ch6ex2}

  \textit{Answer}:
  \begin{equation}
    X(z)=1z^2+2z+5+7z^{-1}+z^{-3}
  \end{equation}

\item Sketch equation~(\ref{eq:zt-delta}).\label{it:ch6ex3}

  \textit{Answer}: In a magnitude ($|\mathcal{X}(\omega)|$)
  vs. frequency plot, it is a horizontal line of value one.

\item Compute the z-transform and frequency content for the signal
  $\delta[n+n_0]$.\label{it:ch6ex4}

  \textit{Answer}:
  \begin{align*}
    \Delta(z) &= \sum_{k=-\infty}^{\infty} \delta[k+n_0] z^{-k} \\
    &= 1z^{n_0}=z^{n_0}
  \end{align*}
  The convergence region is entire $z$ plane, except $z=\infty$. The
  frequency content is
  $|\mathcal{D}(\hat{\omega})| = |\Delta(e^{j\hat{\omega}})| =
  |e^{jn_0\hat{\omega}}|=1$.

\item What is the derivative of $u[n-k]$ (the unit step at time step
  $k$)?\label{it:ch6ex5}

  \textit{Answer}: $\delta[n-k]$.

\item Show that $e^{j\hat{\omega}/2}-e^{-j\hat{\omega}/2} =
  2j\sin\omega/2$.\label{it:ch6ex6}

  \textit{Answer}: Euler's formula states that $e^{j\theta} =
  \cos\theta + j\sin\theta$. For negative angles, $e^{-j\theta} =
  \cos(-\theta) + j\sin(-\theta) = \cos\theta - j\sin\theta$ (because
  cosine is an even function and sine is an odd function). The
  difference of these two is $e^{j\theta}-e^{-j\theta} =
  2j\sin\theta$; substitute $\theta=\hat{\omega}/2$ to finish up.

\item Prove that $e^{-j\pi/2}=-j$.\label{it:ch6ex7}

  \textit{Answer}: Using Euler's formula,
  $e^{-j\pi/2}=\cos\pi/2-j\sin\pi/2=0-j \times 1=-j$.

\item Determine if $u[n] \ast H \ne H$ is true, where $h[n] = n$,
  $n=0,1,2,\ldots$ (a \emph{ramp}).\label{it:ch6ex8}

  \textit{Answer}: Yes, it is true:
  \begin{align*}
    u[n] \ast H &= \sum_{k=0}^{n}1 (n-k) \\
    &= \sum_{k=0}^{n}n -\sum_{k=0}^{n} k \\
    &= n^2 - \frac{n(n+1)}{2} \\
    &= \frac{n(n-1)}{2} \\
    &\ne n
  \end{align*}

\item Compute $u[n] \ast u[n]$.\label{it:ch6ex9}

  \textit{Answer}:
  \begin{equation}
    u[n] \ast u[n]=\sum_{k=0}^{n}1=n
  \end{equation}
  So $u[n] \ast u[n] \ne u[n]$.

\item How does the MATLAB function \verb|conv| deal with boundary
  conditions?\label{it:ch6ex10}

  \textit{Answer}: It zero pads.

\item Use MATLAB to compute the convolution $e^{-n} \ast e^{-n}$ and
  plot the result.\label{it:ch6ex11}

  \textit{Hint}: use \texttt{conv} and \texttt{plot}.

\item Prove the scaling property of the z-transform; that is, if
  \begin{equation*}
    x[n]\stackrel{\mathbf{Z}}{\longleftrightarrow} X(z)
  \end{equation*}
  then 
  \begin{equation*}
    a^nx[n]\stackrel{\mathbf{Z}}{\longleftrightarrow} X(a^{-1}z)
  \end{equation*}\label{it:ch6ex12}

  \textit{Answer}:
  From~(\ref{eq:zt}), 
  \begin{align*}
    Z\left\{a^nx[n]\right\} &= \sum_{n=-\infty}^{\infty} a^nx[n]z^{-n}
    &= \sum_{n=-\infty}^{\infty}x[n](a^{-1}z)^{-n} \\
    &= X(a^{-1}z)
  \end{align*}

\end{enumerate}

\section{Chapter~\ref{ch:fb-filters}: Feedback Filters}
\label{sc:ch4ex}

\begin{enumerate}

\item Derive equation~(\ref{eq:fb-band2b})
  from~(\ref{eq:fb-band2a}).\label{it:ch4ex1}

  \textit{Answer}: Start
  from~(\ref{eq:fb-band2a}):
  \begin{align*}
    2(1-2R+R^2) &= 1-2R\cos\hat{\omega}_B + R^2 \\
    1-4R+r^2    &= -2R\cos\hat{\omega}_B
  \end{align*}
  so 
  \begin{align*}
    \cos\hat{\omega}_B &= -\frac{1}{2R}(1-4R+R^2) \\
    &= 2-\frac{1}{2}(R+\frac{1}{R})
  \end{align*}

\item In the situation where the sampling rate is 44,100Hz and the
  desired bandwidth is 20Hz, $R$ in~(\ref{eq:calc-bw}) is 0.998575.
  Solve for $R$ the situation where the desired bandwidth is 200Hz. Is
  it true that when $R$ is far away from one, $B$ grows
  large?\label{it:ch4ex2}

  \textit{Answer}: $R=1-\pi(200/44100)=0.9858$. Yes.

\end{enumerate}



\section{Chapter~\ref{ch:fft}: Spectral Analysis}
\label{sc:ch7ex}

\begin{enumerate}

\item Show which frequencies will be equal for a spectrum
  with:\label{it:ch5ex4}
  \begin{enumerate}
  \item 16 points.

    \textit{Answer}:
    \begin{equation*}
      \underset{\underset{\text{DC}}{\uparrow}}{0} \quad \underbrace{1
        \quad \overbrace{2 \quad \underbrace{3 \quad \overbrace{4
              \quad \underbrace{5 \quad \overbrace{6 \quad
                  \underbrace{7 \quad 8 \quad 9} \quad 10} \quad 11}
              \quad 12} \quad 13} \quad 14} \quad 15}
    \end{equation*}

  \item 15 points.

    \textit{Answer}:
    \begin{equation*}
      \underset{\underset{\text{DC}}{\uparrow}}{0} \quad \underbrace{1
        \quad \overbrace{2 \quad \underbrace{3 \quad \overbrace{4
              \quad \underbrace{5 \quad \overbrace{6 \quad
                  \underbrace{7 \quad 8} \quad 9} \quad 10} \quad 11}
            \quad 12} \quad 13} \quad 14}
    \end{equation*}
  \end{enumerate}

\item Prove that the DFT of $x[n]$ for any $n=m$ and $N=1$ is
  $x[m]$.\label{it:ch5ex5}

  \textit{Answer}:
  \begin{equation*}
    X[k] = \sum_{n=m}^{m}x[n]e^{-jnk2\pi/1}=x[m]e^{jmk2\pi}=x[m]
  \end{equation*}
  because $k, m$ are integers and so $e^{jmk2\pi}=1$.

\item Perform step-by-step division for the example given in
  table~\ref{tb:fft-bitrevers} to prove the final result is equal to the
  bit-reversed input.\label{it:ch5ex6}

  \textit{Answer}: The answer given in decimal
  is in the following table:
  \begin{center}
    \begin{tabular}{|c|c|c|c|} \hline
      Input & N/2 & N/4 & N/8\\ \hline\hline
      0 & 0 & 0 & 0\\ \cline{4-4}
      1 & 2 & 4 & 4\\ \cline{3-4}
      2 & 4 & 2 & 2\\ \cline{4-4}
      3 & 6 & 6 & 6\\ \cline{2-4}
      4 & 1 & 1 & 1\\ \cline{4-4}
      5 & 3 & 5 & 5\\ \cline{3-4}
      6 & 5 & 3 & 3\\ \cline{4-4}
      7 & 7 & 7 & 7\\ \hline
    \end{tabular}
  \end{center}

\item Perform the 4-point FFT of the signal $x[n] = \{1, 2, 3, 4\}$ by
  hand.\label{it:ch5ex6.5}

  \textit{Answer:} We first do the bit reversal to get:
  \begin{center}
    \begin{tabular}{|c|c|c|c|c|} \hline
      \multicolumn{2}{|c|}{Input} & 
      \multicolumn{3}{c|}{Bit-Reversed Result} \\ \hline
      Decimal & Binary & Binary & Decimal & Value \\ \hline\hline
      0 & 00 & 00 & 0 & 1\\
      1 & 01 & 10 & 2 & 3\\
      2 & 10 & 01 & 1 & 2\\
      3 & 11 & 11 & 3 & 4\\ \hline
    \end{tabular}
  \end{center}

  Each row of this table corresponds to a 1-point FFT. Starting with
  $N=2$, we combine neighbors to form the two-point FFTs according to
  equations~(\ref{eq:2fft-even}) and~(\ref{eq:2fft-odd}):
  \begin{center}
    \begin{tabular}{|c|c|c|}\cline{2-3}
      \multicolumn{1}{c|}{}                 & $k$ & $X[k]$ \\ \hline
      \multirow{2}{*}{\rotatebox{90}{even}} & 0   & 4  \\
      & 1   & -2  \\ \hline
      \multirow{2}{*}{\rotatebox{90}{odd}}  & 0   & 6  \\
      & 1   & -2  \\ \hline
    \end{tabular}
  \end{center}
  
  We now have a 2-point ``even'' FFT and a 2-point ``odd'' FFT.
  We can combine corresponding elements of this using $k=\{0, 1, 2,
  3\}$ and $N=4$; the four equations are:
  \begin{align*}
    X[0] &= X[0]^{\mathit{even}} + e^{-j(0)2\pi/4}X[0]^{\mathit{odd}} \\
    &= X[0]^{\mathit{even}} + X[0]^{\mathit{odd}} \\
    X[1] &= X[1]^{\mathit{even}} + e^{-j(1)2\pi/4}X[1]^{\mathit{odd}} \\
    &= X[1]^{\mathit{even}} + e^{-j\pi/2}X[1]^{\mathit{odd}} \\
    &= X[1]^{\mathit{even}} - j X[1]^{\mathit{odd}} \\
    X[2] &= X[0]^{\mathit{even}} + e^{-j(2)2\pi/4}X[0]^{\mathit{odd}} \\
    &= X[0]^{\mathit{even}} + e^{-j\pi}X[0]^{\mathit{odd}} \\
    &= X[0]^{\mathit{even}} - X[0]^{\mathit{odd}} \\
    X[3] &= X[1]^{\mathit{even}} + e^{-j(3)2\pi/4}X[1]^{\mathit{odd}} \\
    &= X[1]^{\mathit{even}} + e^{-j3\pi/2}X[1]^{\mathit{odd}} \\
    &= X[1]^{\mathit{even}} + j X[1]^{\mathit{odd}}
  \end{align*}
  And so the FFT values (and the \emph{power spectrum}, $|X[k]|^2$) are:
  \begin{center}
    \begin{tabular}{|c|c|c|} \hline
      $k$ & $X[k]$ & $|X[k]|^2$ \\ \hline
      0   & 10      & 100 \\
      1   & -2 + j2 & 8 \\
      2   & -2      & 4 \\
      3   & -2 - j2 & 8 \\ \hline
    \end{tabular}
  \end{center}


\item Fill in the steps leading from~(\ref{eq:usefft-rWk1})
  to~(\ref{eq:usefft-rWk2}).\label{it:ch7ex1}

  \textit{Answer}:
  Equation~(\ref{eq:usefft-rWk1}) is a geometric series with common
  ratio $e^{-j\hat{\omega}_k}$, so
  \begin{displaymath}
    W(\hat{\omega}_k)=\frac{1-e^{-j\hat{\omega}_kN}}{1-e^{-j\hat{\omega}_k}}
  \end{displaymath}
  Using Euler's formula, the numerator becomes
  \begin{equation*}
    1-e^{-j\hat{\omega}_kN} =
    e^{-j\hat{\omega}_kN/2}(e^{j\hat{\omega}_kN/2}-e^{-j\hat{\omega}_kN/2})
    =2j e^{-j\hat{\omega}_kN/2}\sin(\hat{\omega}_kN/2)
  \end{equation*}
  Similarly, the denominator is $2j
  e^{-j\hat{\omega}_k/2}\sin(\hat{\omega}_k/2)$, so
  \begin{displaymath}
    W(\hat{\omega}_k)=\frac{\sin(\hat{\omega}_kN/2)}{\sin\hat{\omega}_k/2}
    e^{-j\hat{\omega}_k(N-1)/2}
  \end{displaymath}
  which is~(\ref{eq:usefft-rWk2}).

\item Plot the Hann and Hamming windows in the time domain and compare
  their shapes.\label{it:ch7ex2}

  \textit{Hint}: Use the MATLAB built-in commands
  \texttt{hamming()} and \texttt{hanning()}.

\end{enumerate}

\section{Chapter~\ref{ch:compression}: Compression}
\label{sc:ch8ex}

\begin{enumerate}

\item If a rate of 44,100 samples/second at 16bits/sample, what is
  the digital data rate in bits/second?\label{it:ch8ex1}

  \textit{Answer}: 705.6kb/s.

\item If we are digitizing high-quality video --- 1k x 1k
  pixels/frame, 30 frames/sec, 24 bits/pixel --- what is the bit
  rate?\label{it:ch8ex2}

  \textit{Answer}: 755Mb/s.

\item If a signal is sent in which all samples have the same value,
  what is the information content in bits (ignoring the first
  sample)?\label{it:ch8ex3}

  \textit{Answer}: Since you can predict all subsequent signals
  with 100\% accuracy, no additional information is sent after the first
  sample (this assumes infinite signal length; otherwise, there is
  additional information --- the number of samples).

\item What kind of signal would have maximum information
  content?\label{it:ch8ex4}

  \textit{Answer}: It would have to be a signal in which the
  next sample could never be predicted at better than chance, regardless
  of the number of previous samples used as ``clues'' to the next
  sample's value. So, for example, if there were 8 bits/sample, the
  chance of predicting the next signal would have to be 1/256. Such an
  unpredictable signal is called \emph{stochastic}, or \emph{random}. In
  multimedia terms, \emph{noise}.

\item Can you give an example of an application which would demand
  symmetric coding?\label{it:ch8ex5}

  \textit{Answer}: Video conferencing. Both
  ends typically have the same hardware, both ends must perform both
  encoding and decoding, and both operations must be done in real
  time.

\item Can you give an example of an application which could allow
  asymmetric coding?\label{it:ch8ex6}

  \textit{Answer}: Video broadcasting. If the
  broadcast is not live, then large computers can be allowed long times
  to optimize a recording.  Even in a live broadcast, the studio can
  invest more money in encoding equipment than the viewer in decoding
  equipment (TVs or computers).

\item Two example codes with the prefix property are given in
  Table~\ref{tb:prefix}. Decoding code 1 is easy, as we can just read
  three bits at a time (for example, decode
  ``001010011'').\label{it:ch8ex7}

  \textit{Answer}: ``2, 3, 4''.

\item What would the symbol sequence be for
  ``01000001000''?\label{it:ch8ex8}

  \textit{Answer}: ``3141''.

\end{enumerate}


\section{Chapter~\ref{ch:audio-video}: Audio \& Video Compression and Coding}
\label{sc:ch9ex}

\begin{enumerate}

\item For almost all versions, the input signal
  is assumed to be 20kHz. What is the minimum sampling rate for such a
  signal?\label{it:ch9ex1}

  \textit{Answer}: 40kHz.

\item Question: under what conditions is it acceptable to have
  greater coder complexity?\label{it:ch9ex2}

  \textit{Answer}: When coding is done
  once and not in real time, or is done by someone with a lot of money,
  like a TV station.

\end{enumerate}


\section{Chapter~\ref{ch:review}: Review and Conclusions}
\label{sc:ch10ex}

\begin{enumerate}

\item Question: what's wrong with low-pass filters with abrupt
  cutoffs (sometimes called \emph{brick wall} filters)?\label{it:ch10ex1}

  \textit{Answer}: Phase distortion; filters with steep cutoffs introduce large,
  frequency dependent delays, or phase shifts.

\end{enumerate}
