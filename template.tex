% -*-LaTeX-*-
\documentclass[12pt]{article}

\usepackage{times,fullpage}

%\usepackage{epsfig}

% Math and logic
\usepackage{amsmath}
\newcommand{\pred}[1]{\ensuremath{\textit{#1}}}
\newcommand{\const}[1]{\ensuremath{\textsf{#1}}}

%\usepackage{algorithmic}
%\usepackage[part]{algorithm}

% Sidebars implemented with picinpar.  Use for defining
% important terms, mentioning important points in the text, indicating 
% the current reading assignment, links to useful resources, etc.
\usepackage{picinpar}
\newcommand{\sidebar}[1]{{\fbox{\parbox{0.4\textwidth}{\small #1}}}}

% Number figures within the lesson
\setcounter{part}{1}
\newcommand{\thepart}{\arabic{part}}
\renewcommand{\thefigure}{\thepart.\arabic{figure}}
\renewcommand{\thetable}{\thepart.\arabic{table}}
\renewcommand{\theequation}{\thepart-\arabic{equation}}

\title{Lesson~\thepart: Name}

\begin{document}
\part*{Lesson~\thepart: Name}

% Use the starred versions of section, subsection, etc.
\section*{Introduction}

% The following is an example of a sidebar with paragraph text
% wrapping around it. Replace the material inside the ``sidebar''
% command with the stuff you want.  Remember that the _entire_
% paragraph needs to be within the window environment. The first
% optional argument to the window environment is the number of lines
% in the paragraph before the window is cut out.  Let's make this 3
% for all but the first paragraph in a page; change it to 0 for others.
\begin{window}[3,r,%
\sidebar{\textbf{Required Reading:}\\
List what's required.},{}]
What is covered by this lesson? The rest of this is just filler to
make the paragraph long enough. dkfjdkj dfdkjfkdj ldjfld dflkjd
dkfjdkj dfdkjfkdj ldjfld dflkjd dkfjdkj dfdkjfkdj ldjfld dflkjd
dkfjdkj dfdkjfkdj ldjfld dflkjd dkfjdkj dfdkjfkdj ldjfld dflkjd
dkfjdkj dfdkjfkdj ldjfld dflkjd dkfjdkj dfdkjfkdj ldjfld dflkjd
dkfjdkj dfdkjfkdj ldjfld dflkjd dkfjdkj dfdkjfkdj ldjfld dflkjd
dkfjdkj dfdkjfkdj ldjfld dflkjd dkfjdkj dfdkjfkdj ldjfld dflkjd
dkfjdkj dfdkjfkdj ldjfld dflkjd dkfjdkj dfdkjfkdj ldjfld dflkjd
dkfjdkj dfdkjfkdj ldjfld dflkjd dkfjdkj dfdkjfkdj ldjfld dflkjd
dkfjdkj dfdkjfkdj ldjfld dflkjd dkfjdkj dfdkjfkdj ldjfld dflkjd 
\end{window}

Objectives --- what the student will be able to do after this lesson.

\section*{Topic}

\begin{figure}
\caption{Example figure caption.}
\end{figure}


Let's have self-test exercises, where appropriate, in each section,
rather than a bunch at the end.

\section*{Assignment~\thepart}

\end{document}
