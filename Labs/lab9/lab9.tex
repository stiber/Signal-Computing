%&LaTeX

\section{Audio \& Image Compression}

\subsection{Lab Background: sound and images in MATLAB}

MATLAB has a number of functions that can be used to read and write
sound and image files as well as manipulate and display them. See the
\texttt{help} for each function for details. Ones that you'll likely
find useful are:
\begin{description}
\item[audiorecorder] Perform real-time audio capture. This may or may
  not work on your system; it is critically dependent on your hardware
  and MATLAB's support thereof. You may find it simpler to record and
  edit audio files with other software and then save it as
  \texttt{.wav} files to load into MATLAB (yes, that's right, MATLAB
  won't read \texttt{mp3} files).
\item[sound] Play vector as a sound. This allows you to create
  arbitrary waveforms mathematically (e.g., individual sinusoids, sums
  of sinusoids) and then play them through your speakers. This is the
  function to use if you're values are scaled within the range of -1
  to +1.
\item[soundsc] This function works like \texttt{sound}, but first it
  scales the vector values to fall within the range of -1 to +1. Much
  of the time, you won't have your waveforms pre-scaled, and you'll
  use this function.
\item[imread] Read image from graphics file. MATLAB's image file
  support is much better than its audio file support. This function
  (and \texttt{imwrite}) supports many file types.
\item[imwrite] Write image to graphics file.
\end{description}

Within MATLAB, audio is a 1-D vector (stereo is $n \times 2$, but we
won't deal with stereo) and grey-scale images are 2D arrays (color
images are 3-D arrays). For simplicity's sake, make sure any sound
files you use are mono.

\subsubsection{Data Types}

MATLAB supports a number of data types, and the type that the file I/O
functions return often depends on the kind of file read. Regardless,
almost all of the MATLAB functions you'll use to process data return
doubles. For example, consider the following commands that read a
color image, convert it to grey-scale, and display it:
\begin{verbatim}
>> A = imread('/tmp/ariel.jpg');
>> B = mean(A,3);
>> imagesc(B)
>> colormap(grey)
>> axis equal
>> whos
  Name      Size                           Bytes  Class

  A       100x55x3                         16500  uint8 array
  B       100x55                           44000  double array
\end{verbatim}
The (color) image is read into \texttt{A}, which, as you can see from
the output of the \texttt{whos} command, is a 100x55x3 unsigned, 8-bit
integer array (the third dimension for the red, green, and blue image
color components). I convert the image to grey-scale by taking the
mean of the three colors at each pixel (the overall brightness),
producing a \texttt{B} array that is \texttt{double}. The
\texttt{imagesc} function displays the image, the \texttt{colormap}
function determines how the array values translate into screen colors,
and the \texttt{axis equal} command ensures that the pixels will be
square on the screen.

Much of the time, we will just perform our calculations using
\texttt{double} data types. However, we can use functions like
\texttt{uint8} to convert our data. Note that the sound playing
functions don't seem to like integer vectors.

At the very least, you can get real audio files from
\url{http://faculty.washington.edu/stiber/pubs/Signal-Computing/}; I assume that
you'll have no trouble locating interesting images (please keep your
work G-rated). Don't use ones that are too big; there will be a 2MB
size limit on E-Submit for your entire compressed file.

Since we will \emph{not} be using WebQ for this lab, write answers to
any questions in the steps below in a file \texttt{answers.txt},
\texttt{answers.doc}, or something similarly named.

\subsection{Lossless image coding}

The simplest way to compress images is \emph{run-length coding} (RLE),
a form of repetitive sequence compression in which multiple pixels
with the same values are converted into a count and a value. To
implement this, we need to reserve a special image value --- one that
will never be used as a pixel value --- as a flag to indicate that the
next two numbers are a (count, value) pair, rather than just a couple
pixels. Let's apply RLE to three different kinds of images: color
photographs, color drawings, and black-and-white images (e.g., a scan
of text). You can choose images you like, or get these from the book
web site:
\url{http://faculty.washington.edu/stiber/pubs/Signal-Computing/ariel.jpg}
(color photo),
\url{http://faculty.washington.edu/stiber/pubs/Signal-Computing/cartoon.png}
(color drawing), and
\url{http://faculty.washington.edu/stiber/pubs/Signal-Computing/text.png}
(black-and-white).

\paragraph{Step 1.1} Write a MATLAB script to read an image in,
convert it to grey-scale if the image array is 3-D, and scale the
image values so they are in the range [1, 255] (note the absence of
zero). Save your script as \texttt{step11.m}.  Verify that this has
worked by using the \texttt{min} and \texttt{max} functions. Convert
the results to \texttt{uint8} and display each using
\texttt{imagesc}. Save the resulting images as \texttt{step11a.jpg},
\texttt{step11b.jpg}, and \texttt{step11c.jpg}. If you used your own
images, name them \texttt{step11a-in.jpg}, \texttt{step11b-in.jpg},
and \texttt{step11c-in.jpg}.

\paragraph{Step 1.2} At this point, you should have in each case a 2-D
array with values in the range [1, 255] inclusive. To simplify
matters, we will treat each array as though it were
one-dimensional. This is easy in MATLAB, as we can index a 2-D array
with a single index ranging from 1 to $N \times M$ (the array
size). Write a RLE function (save it as \texttt{RLE.m}) that takes in
a 2-D array and scans it for runs of pixels with the same value,
producing a RLE vector on its output. When fewer than four pixels in a
row have the same value, they should just appear in the vector. When
four or more (up to 255) pixels in a row have the same value, they
should be replaced with three vector elements: a zero (indicating that
the next two elements are a run code, rather than ordinary pixel
values), a count (should contain 4 to 255), and the pixel value for
the run. Verify that your RLE function works by implementing a RLD
function (RLE decoder, saved as \texttt{RLD.m}) that takes in a RLE
vector, $N$, and $M$ and outputs a 2-D $N \times M$ array. The RLD
output should be identical to the RLE input (subtracting them should
produce an array of zeros); verify that this is the case. For each
image type, compute the compression factor by dividing the number of
elements in the RLE vector by the number of elements in the original
array. What compression factors do you get for each image?

\subsection{Lossy audio coding: DPCM}

In \emph{differential pulse-code modulation} (DPCM), we encode the
differences between signal samples in a limited number of bits. In
this section, you'll take an audio signal, apply DPCM with differing
numbers of bits, see how much space is saved, and hear if and how the
sound is modified. A slight complicating factor is that all of the
data will be represented using \texttt{double}, however, we will limit
the values that are stored in each double to integers in the range
$[0, 2^{\mathrm{bits}}-1]$ (where ``bits'' is the number of bits we're
using).

\paragraph{Step 2.1} Find a sound to work with; you can use
\url{http://faculty.washington.edu/stiber/pubs/Signal-Computing/amoriole2.mat}
if you like. Likely, it will have values that are not integers;
convert the values to 16-bit integer values by scaling (to $[0,
2^{16}-1]$) and rounding (reminder: the vector's type will still be
\texttt{double}; we're just changing the \emph{values} in each element
to be integers in that range). Verify that this conversion produces no
audible change in the sound. What is the MATLAB code to do this
initial quantization?

\paragraph{Step 2.2} Now write a DPCM function, saved as
\texttt{DPCM.m}, that takes the sound vector and the number of bits
for each difference and outputs a DPCM-coded vector. Note that the
MATLAB \texttt{diff} function will compute the differences between
elements of a vector. Your DPCM function should:
\begin{enumerate}
\item compute the differences between the samples,
\item limit each difference value to be in the range
  $[-2^{\mathrm{bits}-1}-1, 2^{\mathrm{bits}-1}-1]$, producing a
  quantized vector and a vector of ``residues'' (you will generate two
  vectors). For each quantized difference, the ``residues'' vector
  will be zero if the difference, $\Delta x_i$, is within the above
  range. Otherwise, it will contain the amount that $\Delta x_i$
  exceeds that range (i.e., the difference between $\Delta x_i$ and
  its quantized value, either $-2^{\mathrm{bits}-1}-1$ or
  $2^{\mathrm{bits}-1}-1$).
\item ``make up for'' each nonzero value in the ``residues''
  vector. For each such value, scan the quantized difference vector
  from that index onward, and modify its entries, up to the
  quantization limits above, until all of the residue has been ``used
  up''.
\item The final result is a single coded vector that your function
  should return.
\end{enumerate}

For example, let's say that we're using 4 bit DPCM and that some
sequence of differences is $(\Delta x_1=10, \Delta x_2=5, \Delta
x_3=-3)$. The range of quantized differences is $[-7, +7]$, and so the
quantized differences are $(7, 5, -3)$ and the residues are $(3, 0,
0)$. Since the first residue is nonzero, we proceed to modify quantized
differences starting with the second one, until we've added three to
them. The resulting final quantized differences are $(7, 7, -2)$.

\paragraph{Step 2.3} Implement an inverse DPCM function IDPCM, saved
as \texttt{IDPCM.m}, that takes the first value from the original
sound vector and the DPCM vector and returns a decoded sound
vector. There's no way for us to know where the losses in the encoding
occurred, so we just use the DPCM values as differences. Note that the
MATLAB \texttt{cumsum} function computes a vector with elements being
the cumulative sum of the elements of its input vector, and that you
can add a constant to all of the elements of a vector using the normal
addition operator.

\paragraph{Step 2.4} Test your DPCM and IDPCM functions by coding your
sound using 15, 14, 12, and 10 bit differences. At what point does the
coding process produce noticeable degradation (on cheap computer
speakers)? Save your original sound vector as \texttt{step24-in.mat}
and \texttt{step24-in.wav} and the IDPCM output sounds as
\texttt{step24-15.mat} and \texttt{step24-15.wav},
\texttt{step24-14.mat} and \texttt{step24-14.wav}, \texttt{step24-12.mat}
and \texttt{step24-12.wav}, and \texttt{step24-10.mat} and
\texttt{step24-10.wav}. See the MATLAB \texttt{save} command for how to
save individual variables in \texttt{.mat} files. For extra credit,
investigate how few bits you can use to encode the sound and still
detect some aspect of the original sound. Is this surprising to you?

\subsection{Lossy image coding: JPEG}

\paragraph{Step 3.1} In this sequence of steps, we will use
frequency-dependent quantization, similar to that used in JPEG, to
compress an image. Start with your grey-scale, continuous-tone image
from step 1.1 (if you used a color image, convert it to grey-scale as
you did in step 1.1). The MATLAB image processing or signal processing
toolboxes are needed to have access to DCT functions, so we'll use the
\verb|fft2()| and \verb|ifft2()| functions instead. To do a basic test
of these functions, write a script that loads your image, converting
it to grey-scale if necessary, and then computes its 2D FFT using
\verb|fft2()|. The resulting matrix has complex values, which we will
need to preserve. Display the magnitude of the FFT (remember to use
the \verb|abs| function to get the magnitude of a complex number)
using \verb|imagesc()|. Do this for each image type.  Can you relate
any features in the FFT to characteristics in the original image?

\paragraph{Step 3.2} Use \verb|ifft2()| to convert the FFT back and
plot the result versus the original greyscale image (use
\verb|imagesc()|) to check that everything is working fine. Analyze
the difference between the two images (i.e., actually subtract them)
to satisfy yourself that any changes are merely small errors
attributable to finite machine precision). Repeat this process for the
other images.

\paragraph{Step 3.3} Let's quantize the image's spectral
content. First, find the number of zero elements in the FFT, using
something like \verb|origZero=length(find(abs(a)==0));|, where
\verb|a| is the FFT. \emph{Remember to exclude the DC value in the
  \texttt{fft2} output in figuring out this range.} Then, zero out
additional frequency components by zeroing out all with magnitudes
below some threshold. You'll want to set the threshold somewhere
between the min and max magnitudes of \verb|a|, which you can get as
\verb|mn=min(min(abs(a)));| and \verb|mx=max(max(abs(a)));|. Let's
make four tests, with thresholds 5\%, 10\%, 20\%, and 50\% of the way
between the min and max, i.e., \verb|th=0.05*(mx-mn)+mn;|. Zero out
all FFT values below the threshold using something like:
\begin{verbatim}
b = a;
b(find(abs(a)<th)) = 0;
\end{verbatim}
You can count the number of elements thresholded by finding the number
of zero elements in \verb|b| at this point and subtracting the number
that were originally zero (i.e., \verb|origZero|). This is an estimate
of the amount the image could be compressed with an entropy
coder. Express the number of thresholded elements as a fraction of the
total number of pixels in the original image and make a table or plot
of this value versus threshold level.

\textit{Extra credit.} Note that the FFT may have very high values for
just a few elements, and low values for others. You might plot a
histogram to verify this. Not including the DC value likely will
eliminate the highest value in the FFT. However, some of the other
values may still be large enough to produce too large of a range. Can
you suggest an approach that will take this into account? How does the
JPEG algorithm deal with or avoid this problem?

\paragraph{Step 3.4} Now we will see the effect of this thresholding
on image quality. Convert the thresholded FFT back to an image using
something like \verb|c = abs(ifft2(b));|. For each type of image and
each threshold value, plot the original image and the final processed
image. Compute the mean squared error (MSE) between the original and
reconstructed image (mean squared error for matrices can be computed
as \verb|mean(mean((a-c).^2))|).  What can you say about the effects
on the image and MSE?  Collect your code together as a script to
automate the thresholding and reconstruction, so you can easily
compute MSE for a number of thresholds. Plot MSE vs. threshold
percentage (just as you plotted fraction of pixels thresholded
vs. threshold in step 3.3).

% LocalWords:  WebQ MATLAB DSP
