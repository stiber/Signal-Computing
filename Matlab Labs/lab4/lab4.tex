%&LaTeX

\section{Feed It Forward}

This lab covers the basic concepts of filtering and feedforward filters. 
You may have also heard of feedforward filters referred to as 
finite impulse response (FIR) filters. In this lab, we will cover the basic idea
of a filter, its mathematical representation (such as the defining equation,
frequency response, and transfer function), the relationship among
filter coefficients, zero placement, and filter type (low pass, high
pass, band reject), and some basic properties of filters.


\subsection{Overview of Filtering and Matlab}

A \emph{digital filter} is a signal processing operation that can be
  described equivalently by its \emph{defining
  equation}, \emph{transfer function}, or \emph{frequency response}. 
  Each representation completely defines the filter. It is advantageous 
  to use each of the different representations depending on whether you are
  \emph{implementing}, \emph{analyzing}, or \emph{designing} an FIR filter:
\begin{eqnarray*}
  y[n] &=& \underbrace{\sum_{k=0}^M b_k x[n-k]}_{\text{defining equation}} \\ 
  \hline \\
  Y &=& H(z) X\\
  H(z) &=& \underbrace{\sum_{k=0}^M b_k z^{-k}}_{ \text{transfer function} } \\
  \hline \\
  Y &=&  \mathcal{H}({\hat{\omega}}) X \\
  H(e^{j\hat{\omega}})  = \mathcal{H}({\hat{\omega}}) 
  &=& \underbrace{\sum_{k=0}^M b_k e^{-j \hat{\omega}k}}_{\text{frequency response}} 
\end{eqnarray*}

In these equations, $x[n]$ and $y[n]$ are the $n^\mathrm{th}$ samples
from the input and output, respectively, while $X$ and $Y$ represent
the entire input and output signal (all of the samples in the
signal). A $k$-sample time delay of a signal is produced by
multiplication by the delay operator, $z^{-1} = e^{-j
  \hat{\omega}k}$. In all three cases (but most simply for the
transfer function), we can obtain insight into the filter's operation
from the \emph{coefficients}, $b_k$. We can do this by factoring the
transfer function polynomial: its roots are the \emph{zeros} of the
filter and they can be real or complex.  The placement of the zeros in
the complex plane (most usefully expressed in polar coordinates) will
tell us which frequencies are suppressed and to what extent those
frequencies are suppressed (we can calculate each using the angle and
magnitude of the zero, respectively).

Matlab has a modest set of functions related to filtering (a much more
substantial set of tools comes along with the Matlab Signal Processing
Toolbox, but we will confine ourselves here to using core Matlab). The
\verb|filter| function applies a general digital filter to a
signal. For this lab, we will stick using this filter as follows:
\begin{lstlisting}[style=Matlab-editor,basicstyle=\mlttfamily\small]
a = [1];              % This will be relevant later for feedback filters
b = [b0 b1 b2];       % Coefficients (remember, Matlab indices start at 1)
y = filter(b, a, x);  % x is input signal; y is output
\end{lstlisting}
This allows us to define the $b_k$ coefficients for a filter and provide
them to the filter function as a vector so that it can filter the
input signal.

We'd also like to specify a feedforward filter by indicating its zero
locations, which we can do by using the \verb|poly| function to
compute the coefficients from a set of roots. So, for example, if we
want a feedforward filter with zeros at $0.9 + 0j$ and $0.75e^{\pm j
\pi/4}$, we can compute the $b_k$ values as:
\begin{lstlisting}[style=Matlab-editor,basicstyle=\mlttfamily\small]
r = [ (0.9 + 0.0*j) (0.75*exp(j*pi/4)) (0.75*exp(-j*pi/4))];
b = poly(r);
a = [1];
y = filter(b, a, x);
\end{lstlisting}
(Where the definition of \verb|r| is written a bit more verbosely than
it has to be.) Note that, from the documentation for \verb|poly|, the
vector of coefficients it produces are ordered from highest to lowest
powers; this corresponds to the same order as the coefficients in the
\verb|b| vector, after dividing by $z^{-M}$, the highest delay
term.

You can visualize the zero locations with the following code:
\begin{lstlisting}[style=Matlab-editor,basicstyle=\mlttfamily\small]
plot(r, 'o')
rectangle('Position', [-1 -1 2 2], 'Curvature', [1 1])
line([-1 1], [0 0], 'Color', [0 0 0])
line([0 0], [-1 1], 'Color', [0 0 0])
axis equal
\end{lstlisting}
Here, we use the unfortunately named \verb|rectangle| function
to draw the unit circle and the \verb|line| function to draw the real
and imaginary axes.

Finally, you can compute and plot the magnitude of the filter's
frequency response pretty directly in Matlab:
\begin{lstlisting}[style=Matlab-editor,basicstyle=\mlttfamily\small]
omegahat = [0: 0.01: pi];   % define the frequency axis
z = exp(j*omegahat);        % define the complex frequency axis
H = polyval(b, z);          % evaluate the transfer function polynomial
plot(omegahat, 20*log10(abs(H)/max(abs(H))));
xlabel('$\hat{\omega}$, radians','Interpreter', 'latex');
ylabel('$|\mathcal{H}(\hat{\omega})|$, dB','Interpreter', 'latex')
\end{lstlisting}
You'll note that the code above plots the ratio of the magnitude of
the frequency response to the maximum value of that magnitude. This is
done because it is convention to ignore whether a filter amplifies a
signal overall; what we are concerned with are the relative amounts
that different frequencies are passed or blocked.


Remember, we can define a specific filter using either of the methods
above. Sometimes it is easier to understand the filter using zeros and
sometimes it is easier to use the coefficients directly.  Either
method can be used to represent the same filter, and we can go back
and forth with the \verb|poly| and \verb|roots| functions (and back
and forth between polar and rectangular representations of complex
numbers with the \verb|abs|, \verb|angle|, and \verb|exp| functions).


\subsubsection{From Filter Coefficients to Transfer Function and Frequency Response}

Given the coefficients of an FIR filter we can solve for the zero
locations and the frequency response.  For example the two-point
averaging system is given by:
\begin{equation}
y[n] = \frac{1}{2}x[n] + \frac{1}{2}x[n-1]
\label{eq:two-point-filter}
\end{equation}
we can find the transfer function by rewriting the filter using the
delay operator, $z$:
\begin{eqnarray}
  Y    & = & \frac{1}{2} X + \frac{1}{2} z^{-1} X \\
  H(z) & = & \frac{Y}{X} = \frac{1}{2} (1 +  z^{-1})
\end{eqnarray}
If we're interested in the \emph{zero location} we can then multiply
$H(z)$ by $z/z$ to obtain:
\begin{equation}
  H(z) = \frac{\frac{1}{2} (z + 1)}{z}
\end{equation}
The root of the numerator, $z=-1$ is the location of the only zero
(the root(s) of the denominator for an FIR filter are always at $z=0$,
and do not affect the frequency response). We can also derive the
frequency response from this by remembering that $H(e^{j\hat{\omega}})
= \mathcal{H}({\hat{\omega}})$ or that $z=e^{j\hat{\omega}}$. From the
zero location, $z=-1$, we can immediately tell that the frequency
response is zero at $e^{j\hat{\omega}}=-1$ or $\hat{\omega}=\pi$. With
a zero at an angle of $\hat{\omega}=\pi$, this is a low-pass
filter. As a warm-up, use Matlab and the coefficients of
equation~\ref{eq:two-point-filter} to verify the expected zero
placement and frequency response.

\subsubsection{From Zero Placement to Filter Coefficients}

When we are given the zero placement, we can very easily determine the
filter coefficients because those zeros are the roots of a factored
polynomial. For example, given two complex conjugate zeros, $z_1$ and
$z_2$ (i.e., the real parts are equal, $\Real[z_{1}] = \Real[z_{2}] =
\Real[z_{1,2}] $, and the imaginary parts are negatives of one another
$ \Imag[z_{1}] = -\Imag[z_{2}] $ or, equivalently in polar
coordinates, $z_{1} = r e^{j\hat{\omega_0}}$ and $z_{2} = r
e^{-j\hat{\omega_0}}$), the transfer function is:
\begin{eqnarray}
  H(z) & = & (z - z_1)(z - z_2)/z^2 \\
  & = & (z^2 - (z_1 + z_2) z + z_1 z_2)/z^2 \\
  & = & 1 - 2 \Real[{z_{1,2}}] z^{-1} + r^2 z^{-2} \\
  & = & 1 - 2 \Real[r(\cos(\omega_0) \pm j\sin(\omega_0))] z^{-1} + r^2 z^{-2} \\
  & = & \underbrace{1}_{b_0} -\underbrace{2 r\cos(\omega_0)}_{b_1} z^{-1} + \underbrace{r^2}_{b_2} z^{-2}
\end{eqnarray}
At this point, we can rewrite the transfer function as the filter's
generating equation, using the delay operator $z^{-k}$, $y[n] = x[n] -
2 r\cos(\omega_0)x[n-1] + r^2 x[n-2]$. This allows us to read off the
filter coefficients: $b_0 = 1$, $b_1 = -2 r\cos(\omega_0)$, and $b_2 =
r^2$.

\subsection{Frequency Response and Pole-Zero Plots}

\paragraph{Step 1.1} Consider a filter that computes a running average
of three points of our input signal (a \emph{three-point averager}):
\begin{equation}
y[n] = \frac{1}{3} \sum_{k=0}^2 x[n-k] 
     = \frac{1}{3}x[n] + \frac{1}{3}x[n-1] + \frac{1}{3}x[n-2]
\end{equation}

\begin{enumerate}\renewcommand{\theenumi}{\alph{enumi}}
\item Draw a block diagram for this filter.


\item How many zeros will this filter have?


\item Find and sketch the zero locations using pencil and paper, then
  use Matlab to verify this.

\item From the plot of pole locations, sketch the magnitude of the
  frequency response as a function of $\hat{\omega}$ by hand and
  verify this using Matlab. How does the minimum of the magnitude of
  the frequency response relate to the polar representation of the
  zero locations?  What kind of filter would you say this is?

\end{enumerate}


\paragraph{Step 1.2} A \emph{first-difference} filter is an
approximation to a discrete derivative operation. Its defining
equation is:
\begin{equation}
  y[n] = x[n] - x[n-1]
\end{equation}

\begin{enumerate}\renewcommand{\theenumi}{\alph{enumi}}

\item Draw a block diagram for this filter.


\item Derive the transfer function, $H(z)$, for this filter. From
	  this, determine the expression for the frequency response,
	  $\mathcal{H}(\hat{\omega})=H(e^{j\hat{\omega}})$.


\item From the transfer function, determine the filter's zero
	locations and sketch them. Check your results with Matlab.


\item From the zero plot, sketch the magnitude of the filter's
	  frequency response as a function of $\hat{\omega}$. Use Matlab to
	  check your results. What kind of filter would you say this is?
	  

\item Use Matlab to compute this filter's response to the following
	  input. Generate an analog signal that is a sinusoid with
          amplitude of 1, frequency of 2, and duration of 1. Sample
          it at 32 samples/second and quantize it using 16
          bits. What is the digital frequency, $\hat{\omega}$, of this
          $f=2$Hz sinusoid?

\item Produce a figure with two plots: the top should be the
          original digital signal, $X$, and the bottom should be the
          filtered signal, $Y$.

\item Examine the plots of $X$ and $Y$. Note that $Y$ appears
	  to be a scaled and shifted sinusoid of the same frequency as
	  $X$. The exception is the first point, $y[0]$. Explain why $y[0]$ is
	  different (if you are unsure, consider the defining equation
          and the input values to it for $n=0$).


\item Estimate the frequency, amplitude, and phase of $Y$ directly
	  from its plot (ignoring $y[0]$).


\item To compare these measurements to theory, use your
          expression for the filter's frequency response to calculate
          the amplitude and phase at the digital frequency
          $\hat{\omega}$ you determined above. How do these compare to
          what you determined from the Matlab plots?


\end{enumerate}

\paragraph{Step 1.3} Just as we can compute a discrete first
derivative with a first-difference filter, we can compute a discrete
second derivative with a \emph{second difference filter}.

\begin{enumerate}\renewcommand{\theenumi}{\alph{enumi}}
\item Use your expression for the transfer function of the first
	  difference filter and your knowledge that the combined transfer
	  function of two filters cascaded, or connected in series, is the
	  product of their individual transfer functions to determine the
	  transfer function for a second-difference filter.


\item Draw a block diagram for this filter.
	

\item Determine the filter's zero locations and sketch them. Check
	  your results using Matlab.
	  
	  

\item From the zero plot, sketch the magnitude of the filter's
          frequency response as a function of $\hat{\omega}$. Use
          Matlab to check your results. What kind of filter would you
          say this is?


\end{enumerate}


\paragraph{Step 1.4} Consider a feedforward filter with complex
conjugate zeros at $z_{1,2} = -0.5 \pm j 0.5$.

\begin{enumerate}\renewcommand{\theenumi}{\alph{enumi}}
\item Determine the filter coefficients.


\item Use Matlab to plot the frequency response of the filter.

\item What are the effects of the zeros on the frequency response?
	  What kind of filter would you call this?

\end{enumerate}

\subsection{Linearity and Cascading Filters}

\paragraph{Step 2.1} A system is called \emph{linear} if a sum of
different inputs produces an output that is the sum of the outputs for
the inputs taken individually.  Perform a simple test of the linearity
of this filter by doubling the input amplitude in Matlab ($X' = 2X = X
+ X$). How does the new output amplitude compare to the old one?


\paragraph{Step 2.2} In one of the self-test exercises in the class
notes, two filters with transfer functions $H_1(z) = b_0 + b_1z^{-1}$
and $H_2(z) = b'_0 + b'_1z^{-1}$ were connected in series, and it was
shown that they could be connected in either order to produce the same
composite effect (the same overall transfer function). Redo this
exercise using the \emph{defining equations} for the two filters,
i.e., $y_1[n] = F_1(x[n])$ for the filter with transfer function
$H_1(z)$ and $y_2[n] = F_2(x[n])$ for the filter with transfer
function $H_2(z)$. In other words, show that $F_2(F_1(x[n])) =
F_1(F_2(x[n]))$.


\paragraph{Step 2.3} Use Matlab to implement a 50\% duty cycle square
wave with amplitude 1, frequency 2Hz, and duration 1s. Sample and
quantize it appropriately (to make your figures look nicer, feel free
to chose a sampling rate much higher than the minimum). Send the
resultant digital signal through the previously-defined three-point
averager filter, and then the output of that filter through the first
difference filter. Plot the input and output. What does the output of
this combined filter look like?


\paragraph{Step 2.4} Now, switch the order you apply the filters so
that the first difference filter is first and the three-point averager
is second. Plot the input and output. How does the output of this
configuration compare to that of the preceding step? Does this match
what you expected? Why or why not?


% LocalWords:  WebQ MATLAB
