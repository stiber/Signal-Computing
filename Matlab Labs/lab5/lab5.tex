%&LaTeX

\section{Let's Catch Some Z's}

This lab covers the z-transform, used to convert arbitrary digital
signals to the frequency domain. It also exercises the relationship
between a filter's transfer function and impulse response and how the
operations of multiplication and convolution, respectively, can be
used to compute a filter's output.

\subsection{The z-transform, Transfer Function, \& Impulse Response}

A discrete signal $x[n]$ has a z-transform $X(z)$ defined by the
following equation:
\[
X(z)=\sum_{n=0}^{\infty}x[n]z^{-n}
\]
With this definition lets investigate a feed forward filter with ten
coefficients, $\{b_0, b_1,\cdots, b_9\}$.  Recall that the Matlab
\verb|filter| function allows us to specify a filter in terms of its
\emph{coefficients}, but we can also think of it as being defined in
terms of its \emph{transfer function}. Considering the $b_k$
coefficients of the above feed forward filter, the \verb|filter|
function implements the transfer function:
\begin{equation}
  H(z) = \sum_{k=0}^{9} b_k z^{-k}
\end{equation}
In previous labs we have computed the transfer function using the
delays of the \emph{defining function}.  Mathematically, we were
actually taking the z-transform of the \emph{impulse response}!  In
this example, the impulse response is:
\begin{equation}
  h[n] = \sum_{k=0}^{9} b_k \delta[n-k]
\end{equation}
where $\delta[k]$ is the unit impulse and only has a non-zero value at
$k=n$. $H(z)$ and $h[n]$ form a z-transform pair,
$h[n]\xleftrightarrow{z} H(z)$. It should now be obvious why
feedforward filters are also known as finite impulse response filters
--- their impulse response only has a \emph{finite} number of
values. To compute the output, $y[n]$, using the impulse response we
use \emph{convolution}. Namely, we \emph{convolve} the input, $x[n]$,
by the impulse response, $h[n]$,
\begin{equation}
  y[n] = x[n] \ast h[n]  = \sum_{k=0}^{9}x[k]h[n-k]
\end{equation}
And, indeed, Matlab has a \verb|conv| function to do this convolution.
Alternatively, we can compute a filter's output by multiplying the
transfer function by the z-transform of the input to yield the
z-transform of the output:
\begin{equation}
  Y(z) = H(z) X(z)
\end{equation}
From a practical point of view, of course, it makes more sense to
implement a filter in terms of its impulse response. However, for
filters with long impulse responses, it is sometimes more convenient
to represent them mathematically using the transfer function (which we
now know is just the z-transform of the impulse response!).

\subsection{Z-Transforms}

\paragraph{Step 1.1} On paper, compute the z-transform, $X(z)$, of
	\begin{equation}
		x[n] = \left\{
		\begin{array}{ll}
			(-1)^n & n \ge 0 \\
			0 & n < 0
		\end{array}\right.
	\end{equation}
	Note that this is an infinite geometric series.  What are the
        locations of any pole(s) (roots of the denominator polynomial)
        or zero(s) (roots of the numerator polynomial)?


\paragraph{Step 1.2} Evaluate the frequency response of $X(z)$
        from step~1.1, $X(z)\big{|}_{z=e^{j\hat{\omega}}}$, by
        sketching it by hand.  What kind of filter is this?


\paragraph{Step 1.3} Consider the z-transform:
	\begin{equation}
	  X(z) = 1 - 2z^{-1} + 3z^{-3} - z^{-5}  
	\end{equation}
	Write the inverse z-transform, $x[n]$, as a table of values for
	corresponding $n$ values.


\subsection{Impulse Response}

\paragraph{Step 2.1} Consider a filter with a transfer function
	\begin{equation}
	H(z) = 1 + 5z^{-1} - 3z^{-2} + 2.5z^{-3} + 4z^{-8}  
	\end{equation}
	What is the defining equation for this filter, $y[n] = F(x[n])$?


\paragraph{Step 2.2} What is the output sequence of the filter of
	Step~2.1 when the input is $x[n] = \delta[n]$? Verify this
        using Matlab.


\paragraph{Step 2.3} The impulse response of a filter is $h[n]
        = \delta[n] + 2\delta[n-1] + \delta[n-2] - \delta[n-3]$, or equivalently,
        $h[n]=\{1,2,1,-1\}$, $n=\{0, 1, 2, 3\}$. Determine the
        response of the system to the input signal $x[n]=\{1,2,3,1\}$,
        $n=\{0, 1,2,3\}$ by hand. Use Matlab to check your
        results. Include a figure that shows both the input and output
        signals; make sure the reader can clearly see what the signal
        values are (the \verb|stem| plot function should help to
        ensure this is the case).

\paragraph{Step 2.4} Change the input to the filter of
        Step~2.3 to be $\delta[n]$. What are the output values? How do
        they compare to the impulse response? Include plots of the
        filter input and output values in your report.


\paragraph{Step 2.5} Use Matlab to determine the output of the
        filter \{1/3,1/3,1/3\}, $n=\{0,1,2\}$ for the input:
	\begin{equation}
	  x[n] = 4 + \sin[0.25\pi(n-1)] - 3 \sin[(2\pi/3)n]  
	\end{equation}
        Include a listing of your Matlab code and a figure with
        plots of the filter input and output in your report. Is the
        result expected? Why or why not?
	
	

\paragraph{Step 2.6} Create your own Matlab \emph{function},
        \verb|convolution|, to implement a convolution function.  To
        test your function, make sure it works exactly like the Matlab
        \verb|conv| and \verb|filter| functions by providing the same
        input to each and subtracting their outputs. Use the filter
        $\{1/3,1/3,1/3\}$, $n=\{0,1,2\}$ from step 2.5.
	

\subsection{Canceling Sinusoidal Components}

Filters can be designed to cancel sinusoids.  Implement a filter in
Matlab with the following impulse response:
\begin{equation}
h[n] = \delta[n] - 2\cos(\pi/4) \delta[n-1] + \delta[n-2]
\end{equation}

\paragraph{Step 3.1} Plot the frequency response for this filter. What
	are the zero locations?


\paragraph{Step 3.2} Use as an input to this filter the signal
        $x[n] = \sin\hat{\omega}n$, using the two frequencies
        $\hat{\omega} = \pi/2$ and $\hat{\omega} = \pi/4$. You will
        need to choose appropriate analog signals, with convenient
        frequencies and durations, and then sample them appropriately
        so they have the correct digital frequencies. Make sure to
        verify that you get the correct digital frequencies and that
        plots you make are convenient for the reader (for example,
        neither too many nor too few cycles)! Compute the filter in
        Matlab for each of these two inputs, plotting the input and
        output of each. When do you get cancellation?

\paragraph{Step 3.3} Can you modify the filter coefficients to cancel
	the other sinusoid? If so, show your work.




%%% Local Variables: 
%%% mode: latex
%%% TeX-master: t
%%% End: 
