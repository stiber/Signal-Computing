%&LaTeX

\section{Discrete Fourier Transform}

\subsection{Lab Background}
By the end of this lab you should have a firm understanding of how the
Discrete Fourier Transform (DFT) can be implemented exactly using the
Fast Fourier Transform (FFT). In addition you should be able to
identify common problems using the DFT to analyze signals. You will
also be familiar with a new tool, the spectrogram, that uses the DFT
as a function of time.

\subsection{Implementing the DFT}

% TODO: have them work with this block for a good implementation of the J-DSP
Recall that the DFT can be implemented directly from the analysis
equation. For a length $N$ signal $x[n]$,
\begin{align}
X[k]=\sum_{n=0}^{N-1}x[n]e^{-j \frac{2\pi}{N} nk} && \text{for $k = 0, 1, 2, \cdots N-1$}
\label{eq:dft}
\end{align}
The order of the implementation is $O(N)=N^2$. The following java code
can be used to implement the DFT in J-DSP. It is implemented without
any algorithmic speedup (i.e., it exactly mirrors equation
\ref{eq:dft}).  Example of raw DFT implementation:
	\begin{lstlisting}
public class MyFunction1
{
 public void myCode(double[]x1,double[]x2,double[]y1,double[]y2,
 	double[]b1,double[]a1,double[]b2,double[]a2,
 	double para1, double para2, double para3)
  {
  // x1, x2 - input at pin 0 and pin1
  // y1, y2 - output at pin 4 and pin5
  // raw DFT implementation
  double[] yImag;
  double[] yReal;
  yImag = new double[256];
  yReal = new double[256];
  double twoPiOverN = 2*Math.PI/256;
  for( int k = 0 ; k < 256 ; k++)
  {
    yReal[k] = 0;
    yImag[k] = 0;
    for( int n = 0 ; n < 256 ; n++)
    {
      yReal[k] += x1[n]*Math.cos(n*k*twoPiOverN);
      yImag[k] += -x1[n]*Math.sin(n*k*twoPiOverN);
    }
    y1[k] = Math.sqrt(yReal[k]*yReal[k]+yImag[k]*yImag[k]);
  }
 }
}
	\end{lstlisting}
Note that there is not a native support in java for complex numbers so
this arithmetic is written out explicitly in the code above. For
example, the equation $y=x\times e^{a}$ (where x is a real number)
must be explicitly written out using Euler's formula, and the real and
imaginary portions saved in separate variables, $y_{real}=x\times
\cos(a)$ and $y_{imag}=x\times \sin(a)$.

The FFT algorithm, on the other hand, can be used to reduce the
computation time of the DFT to $O(N)=N\log_2 N$ - a significant
speedup for even modest length signals.

\paragraph{Step 1.1}
Write some example code to multiply \emph{two} complex numbers,
$c=a\times b$. Remember to represent the variables by their real and
imaginary parts and save the real and imaginary parts in separate
variables.


\paragraph{Step 1.2} 
Use the \block{UserDefinedFun} block to implement the FFT using loops
(i.e., do not use recursion). The first part of this code performs a
bit reversal on the input array. Use the following code to perform the
bit reversal:
% make more explict, bit rev vrbl
\begin{lstlisting}
// bit reversal
int  index;
int  N = 256;
double[] reversedX1 = new double[256];
// need to shift 32 bit integer to make it an 8 bit reversal
int shift = (32-(int)Math.round(Math.log(N)/Math.log(2))); 
for( int n = 0 ; n < N ; n++){
  index = (Integer.reverse(n))>>>shift; //reverse and then shift down to 8 bit
  reversedX1[n] = x1[index]; 
}
\end{lstlisting}
Then iterate over the array $\log_2 N$ times! Remember to explicitly
carry out any complex arithmetic and take the magnitude of the output
array once the FFT is computed. Include a copy of the java code in
your report.

\paragraph{Step 1.3} 
Check your results in J-DSP using the \block{FFT} block. Your results
should be identical. Take the FFT of a sinusoid with a frequency of
$\pi/4$ radians per second using your FFT implementation and the
\block{FFT} block provided in J-DSP. Include a screenshot of the
output plots.



\subsection{Using the DFT}
% switch over to spectrogram example with DTMF - organize introduction
% for ease of use over multiple steps
\paragraph{Step 2.1} 
Create a sum of two sinusoids in J-DSP. Use the \block{FFT} block to
compute the FFT of the sum and then plot the FFT magnitude. Use a
frequency of $0.13\pi$ and $0.19\pi$ for the two sinusoids. Make sure
that the \option{pulsewidth} for each is set to 256. What does the
result look like? Does this make sense?


\paragraph{Step 2.2} At what index (or indices) does the FFT magnitude
reach its peak value(s)? What frequency (or frequencies) does this
correspond to?


\paragraph{Step 2.3} Change the frequencies of the sinusoids to
$0.13\pi$ and $0.14\pi$. Repeat steps 2.1 and 2.2. Do the results
still make sense?


\paragraph{Step 2.4} Replace the sum of two sinusoids in J-DSP with
the \block{DTMF} block under the \menu{Audio Effects} function
list. Press some of the keys in the \block{DTMF} block. Does the FFT
block show separate frequencies for each button?



\subsection{Spectrograms}
Comparing FFT graphs (as in the step 2.4) can be difficult to do. But
what if we could analyze the frequency content of a signal as a
function of time? That would make it easier to see differences in
frequency if a signal started changing (like a string of DTMF keys
pressed in turn). To do this we will need a new tool called the
\emph{Spectrogram}. The spectrogram is simply an algorithm for
computing the FFT of a signal at different times and plotting them as
a function of time. The spectrogram is computed in the following way:

\begin{enumerate}
\item A given signal is ``windowed.'' This means that we only take a
  certain number of points from the signal (for this example assume we
  are using a window of length 128 points). To start out, we take the
  first 128 points of the signal (points 0 through 127 of the input
  array).
\item Take the FFT of the window and save the it in a separate array.
\item Advance the window in time by a certain number of points. For
  instance we can advance the window by 64 points so that we now have
  a window of indices 64 through 191 from the input signal array.
\item Repeat steps 1-3, saving the FFT of each window, until there are
  no longer any points in the input array.
\item Form a 2-D matrix whose columns are the FFT magnitudes of each
  window (placed in chronological order). In this way, each row
  represents a certain frequency, each column represents a given
  instant in time, and the value of the matrix represents the
  magnitude of the FFT.
\end{enumerate}

The result is called a Spectrogram and is usually displayed as an RGB
image where blue represents small FFT magnitudes and red represents
larger FFT magnitudes (the \emph{Jet} colormap if you are familiar
with color visualizations). There is an art to choosing the correct
parameters of the spectrogram (i.e., window size, FFT size, how many
points to advance the FFT, etc.). Each parameter has tradeoffs for the
time and frequency resolution of the resulting spectrogram. For our
purposes here, we will not be concerned with these tradeoffs. Instead
we will be more interested in getting familiar with analysis using
spectrograms.

\paragraph{Step 3.1} Use the DTMF setup from step 2.4. Instead of
using the \block{FFT} block, use the \block{Spectrogram} block under
the \menu{Statistical DSP} function stack. Keep all parameters in the
\block{Spectrogram} set to the default. Set the \block{DTMF} block to
play five frames per key press. Press the number ``1.'' What does the
spectrogram look like? Are both frequencies present?


\paragraph{Step 3.2} Now set the \block{DTMF} block to
\option{Record}. Also set the \option{Resolution} parameter inside the
\block{Spectrogram} block to 4 or 8 (otherwise you may notice a lag
for updating the spectrogram - it can be considerably CPU intensive
for large input signals). The \option{Resolution} parameter controls
how many points the sliding window advances at each time step (larger
steps mean we take fewer FFTs). Press each key in the \block{DTMF}
block in turn. Does the spectrogram make it easier to judge the
frequency content of the keys? Include a screenshot for your report.

% LocalWords:  WebQ MATLAB DSP
