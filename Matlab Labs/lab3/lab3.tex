%&LaTeX

\section{Hello, Digital!}

In this lab, you will investigate how capturing an analog signal for
computer use --- sampling and quantization --- modifies the signal,
seeing how the choices you make in the parameters for sampling and
quantization affect the quality of the digitized, computer signal.

\subsection{Sampling}

The first step in digitization is \emph{sample and hold}, in which the
continuous analog signal is converted to a discrete-time analog signal
(an analog signal that only changes its value at particular points in
time). You will use the \texttt{samplehold} method to do this:
\begin{lstlisting}[style=Matlab-editor,basicstyle=\mlttfamily\small]
% samplehold Perform a sample and hold function on an AnalogSignal
% Usage:
%  x = obj.samplehold(h)
% where  obj = AnalogSignal
%        h = hold time in sec (sampling interval)
%        x = resultant sampled AnalogSignal
\end{lstlisting}

\paragraph{Step 1.1} Create an analog sine waveform ranging from -5 to
5V with a frequency of 200Hz and a duration of 2 seconds. Produce a
plot with X-axis limits set to make the waveform visible (i.e., don't
just make a 2s plot that tries (and fails) to show 400 cycles of the
sinusoid.

\paragraph{Step 1.2} Use the \texttt{samplehold} method to produce
sampled versions of this signal at 300Hz, 500Hz, 1000Hz, and
2000Hz. Use the Matlab \texttt{subplot} command, and the
\texttt{discreteplot} function provided with this class's Matlab code,
to plot the original and all four sampled signals together. One of the
things that you will note from these plots is that the X axes of these
plots do not have units of continuous time; their units are \emph{not}
seconds. Instead, the X axis values are sample count, starting with
sample 1. You will want these plots to show their signals over the
same time duration, and so you'll need to choose that duration,
then figure out, for each sampled signal, how many samples correspond
to that duration, and then set the X axis limits for each figure so
that the plots show equal durations.

Clearly, the results are not the same, and none look identical to the
original sine wave. What are the two essential pieces of information
about a sine wave that need to be preserved when sampling it? Does it
appear that all sampled versions are equally useful in achieving this?
Why or why not (in other words, your answer to this question should
not be just ``yes'' or ``no'')?

\paragraph{Step 1.3} Let's look at aliasing in a little more detail
and with a lot more numerical precision. You'll recall from the text
that, once we sample a signal, we have limited the range of
frequencies that we can represent in our discrete signal to the range
$0 \leq \hat{\omega} \leq \pi$, corresponding to a range of apparent
frequencies in the physical world of $0 \leq \omega' \leq \omega_s/2$
(or $0 \leq f' \leq f_s/2$). Any frequency in the original signal
above $f_s/2$ will be \emph{aliased} into the range of possible
apparent frequencies. To keep things simple, we'll stick with a
sinusoid; this time, make it 10Hz with an amplitude of -1 to +1 and a
duration of 1s. This will make it easy to count cycles when
plotted. Set up a figure that can hold three plots and plot this
analog signal in the top plot.

\paragraph{Step 1.4} Sample this signal at 25Hz and use the
\texttt{discreteplot} function to plot the sampled signal in the
middle. Sample it at 15Hz and similarly plot that sampled signal at
the bottom.

\paragraph{Step 1.5} Before examining the plots in detail, answer the
following questions: For each of the two sampling frequencies, what is
the range of apparent frequencies that can be represented? For each,
will a sinusoid with $f = 10$Hz be aliased? If so, what will be the
digital frequency and the apparent frequency of a 10Hz sinusoid?

\paragraph{Step 1.6} Examine the plots and count the number of
up-and-down cycles in each. Don't worry that each cycle doesn't look
the same, or that every other cycle seems different; just count
each. You should see 10 cycles in the top graph; how many do you see
in the middle and bottom? How does this compare to the theory you
discussed in the previous step? If there is any discrepancy, explain it.


\subsection{Analog to Digital Conversion}

The last step of digitization is called ``analog to digital
conversion,'' or \emph{quantization}. In this step, the sampled analog
signal is converted to a discrete signal, with values represented by
$b$ bit integers. We will use the \texttt{quant} function provided
with this class's Matlab code to perform this conversion:
\begin{lstlisting}[style=Matlab-editor,basicstyle=\mlttfamily\small]
% QUANT Quantize a sampled (discrete) signal using a prescribed
%       number of bits per point.
% Usage:
%  y = quant(x,nb,out)
% where y = digital signal quantized to 2^(nb) bits resolution
%       x = vertical points of sampled signal
%       nb = number of bits to use per point
%      out = 'raw' means output binary values: 0,...,2^(nb-1)
%      otherwise, set ouput value range = input value range
\end{lstlisting}

\paragraph{Step 2.1} Write a Matlab function that compares two signals
by computing the \emph{signal to noise ratio} (SNR) that results from
changing one into the other (by quantization). Your function should do
this by first computing \emph{root mean squared} (RMS) error between
the two. In this case, you will be comparing a sampled signal with a
quantized version of it.  To do this, you should subtract the
quantized signal from the sampled signal to produce a vector of
differences (errors), then square each difference value using the
MATLAB \verb|.^| operator (to get squared errors), then take the mean
of these squared values using the MATLAB \verb|mean| function (mean
squared error --- a scalar value), and finally take the square root of
that scalar result (root mean squared error). We can compute the RMS
signal in a similar way, by squaring the signal values, getting the
mean of those squared values, and then taking the square root. The
final result should be a single value --- the signal-to-noise ratio
(SNR) in \emph{decibels}. What is in the numerator and what is in the
denominator of this ratio? Note that this is different than what we
did in the textbook, because we are now doing the computation for a
\emph{specific} signal, not just figuring SNR for a possible
\emph{range} of signal values.



  \paragraph{Step 2.2} Use your code to compute the SNR for a
  quantized sinusoid. Generate an analog signal with with a range of 0
  to 5, frequency of 10Hz, and duration 2sec. Sample it at 25Hz. Use
  2, 4, 8, 12, and 16 bits quantization, and plot SNR on the Y-axis
  versus number of quantization bits on the X-axis.

  \paragraph{Step 2.3} Repeat Step 2.2 using a square waveform with
  the same parameters.

  \paragraph{Step 2.4} Repeat Step 2.2 using a triangle waveform with
  the same parameters.

  \paragraph{Step 2.5} As you double the number of bits used in
  quantization, how does the SNR change? How does this compare to what
  your learned from the textbook? Refer to specific features of your
  plots from Steps~2.2--2.4 to justify your answer.
