%&LaTeX

\section{Let's Get Physical}

In this lab, you will use Matlab to explore how a physical signal can
be considered to be composed of a sum of sinusoids --- its
\emph{Fourier series}. You will then investigate how capturing this
signal for computer use --- sampling and quantization --- modifies the
signal. Finally, you will see how the choices you make in the
parameters for sampling and quantization affect the quality of the
digitized, computer signal. You will be using the Matlab
\texttt{AnalogSignal} class and functions at
\url{http://faculty.washington.edu/stiber/pubs/Signal-Computing/}.

\subsection{Beating}

In this section, you will use the Matlab \texttt{AnalogSignal} class
to simulate an analog signal generator. The constructor takes the
following arguments:
\begin{verbatim}
% AnalogSignal(type, amplitude, frequency, dur)
%    Where:
%    type = 'sine', 'cosine', 'square', 'sawtooth', or 'triangle'
%    amplitude = signal amplitude
%    frequency = signal frequency
%    dur = signal duration
\end{verbatim}

Remember that, though an \texttt{AnalogSignal} is simulating an analog
signal, in Matlab all functions are sampled at discrete points in
time. The crufty details of this are hidden away in the class's
implementation. Remember also that you can always plot an
\texttt{AnalogSignal} to see what you have; include such figures in
your report 

\paragraph{Step 1.1} Verify that you can get a triangle wave. What is
the code to generate and plot a triangle wave ranging from -1 to 1
Volts with a frequency of 10Hz and a duration of 1sec?

\paragraph{Step 1.2} Next, verify that you can generate a sine wave of
1 second duration at a frequency of 440Hz, ranging from -1 to +1. Use
the \texttt{AnalogSignal} \texttt{soundsc} method to play this as a
sound. This pitch corresponds to ``standard A'' on a musical scale ---
A above middle C. Use the Matlab \texttt{set(gca, 'XLim',
  [\itshape{Xmin}, \itshape{Xmax}])} function call to set the X axis
limits so that the waveform is apparent (i.e., you're not just
plotting a solid blob).

\paragraph{Step 1.3} Generate sine waves of identical range and
duration, but with frequencies of 442, 444, and 448 Hz. Now, generate
the sums of 440Hz and each of these new signals to generate beating
akin to tuning an instrument against the 440Hz standard. Play each sum
signal; can you hear the beating? Plot each sum signal for its full 1s
duration. The beating ``envelope'' should be obvious. What is the beat
frequency in each case? . How does this relate to the textbook
discussion of beating?


\subsection{Fourier series representation of a physical signal}


\paragraph{Step 2.1} Recall that any periodic signal can be
represented as a sum of harmonic sinusoids.  The amplitude of each
harmonic is known as the Fourier Series. It may at first seem like
sums of sinusoids would be poor approximations of real periodic
signals, but this is not the case. We can illustrate this using a
triangle wave. The formula for synthesis of a triangle wave with
frequency $\omega_0$ is a sum of harmonically related sine waves (its
Fourier series):
  \[
  x(t) = \sum_{k=0}^{\infty}
  \left( 
    \underbrace{ \frac{8}{\pi^2} \frac{(-1)^k}{(2k+ 1)^2} }_{ \text{amplitude} } 
    \underbrace{ \sin((2k+1)\omega_0 t) }_{ (2k+1)^{th}\text{ harmonic} } 
  \right)
  \]
  In this case, in the analog domain, we are dealing with frequencies
  in Hz, and so $\omega_0 = 2\pi f_0$. Notice that the Fourier Series
  of the triangle wave only uses odd harmonics (i.e., the only
  non-zero frequencies are $(2k+1)\omega_0=\omega_0, 3\omega_0,
  5\omega_0 \cdots$). Also notice that resulting wave will have zero
  mean because there is no ``DC'' term (i.e., $2k+1 \neq 0$ for any
  integer k).

  Write a Matlab script that approximates a triangle wave by summing
  together the first 7 harmonics of its Fourier series; plot the
  resultant signal (i.e., use $f_0$, $2f_0$, $3f_0$, $\cdots 7f_0$,
  where $f_0=$10 Hz). How does this signal compare to the triangle
  wave computed directly in the previous step?



\paragraph{Step 2.2} Another way to view a signal is in the
\emph{frequency domain}. For a signal expressed in terms of its
Fourier series, the frequency representation is merely the
coefficients of the harmonics. Write a Matlab function or script to
compute and plot the spectrum of a triangle wave. You may find the
MATLAB function \texttt{stem} useful.  Note that you are \emph{not}
being asked to plot the triangle wave as a function of time; you
should plot the amplitudes of the component sinusoids as a function of
those sinusoids' frequencies (like the vertical lines in textbook
figure~1.12).  Use your code to plot the spectrum of the triangle wave
from the previous step (first 7 harmonics).
