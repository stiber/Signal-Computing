%&LaTeX

\section{To Infinity and Response!}

By the end of this lab you should feel comfortable manipulating and
using feedback filters for simple problems. You should also be
comfortable with the concept of a filter with an infinite impulse
response. All feedback filters have an infinite impulse response
and are also known as IIR filters. Feedback filters use the previous
outputs of the filter, feeding them back to compute the output for the
current sample. The ``fed back'' outputs are weighted by coefficients,
$a_\ell$.

\subsection{A Note About Matlab Filter Coefficients}

Note that the Matlab \verb|filter| function uses \emph{negative}
values for the $a_\ell$ (feedback) coefficients, from the transfer
function. In other words, in the text, a second-order feedback
filter's defining equation might be:
\begin{align}
  y[n] &= a_1 y[n-1] + a_2 y[n-2] + b_0 x[n] \label{eq:def} \\
  y[n] -a_1 y[n-1] - a_2 y[n-2]&=   b_0 x[n]
\end{align}
This yields the transfer function:
\begin{align}
  Y(z)(1 - a_1 z^{-1} - a_2 z^{-2}) &= b_0X(z) \\
  Y(z)/X(z) &= \frac{b_0}{1 - a_1 z^{-1} - a_2 z^{-2}} \\
  H(z) &= \frac{b_0}{1 - a_1 z^{-1} - a_2 z^{-2}} \label{eq:trans} 
\end{align}
The coefficients used by \verb|filter|, rather than being the $a_\ell$
from the defining equation~(\ref{eq:def}) are the \emph{negative}
$a_\ell$ from the transfer function~(\ref{eq:trans}) --- the ratio of
two polynomials. In other words, to properly compute the above filter
in Matlab, you will need to use:
\begin{lstlisting}[style=Matlab-editor,basicstyle=\mlttfamily\small]
a = [1.0 -a1 -a2];
b = [b0];
y = filter(b, a, x);
\end{lstlisting}
Note also that in this example the filter includes the $a[0]$
coefficient (of course, per Matlab one-based indices, as \verb|a(1)|),
which we will always leave as 1.0 (it's the first ``1'' in the
denominator of the transfer function).

And finally, note that one form of the \verb|filter| function takes a
fourth argument, which is the initial conditions for the feedback
delays (used in computing the output values that have delay terms
which come before the first value in the $x$ vector). These default to
all zero if not specified. 

\subsection{Feedback Filters as Recurrence Relations}

You may notice that the defining equation for a feedback filter is in
the form of a recurrence relation. In fact, we can use a feedback
filter to implement a recurrence relation if we set the input to be
an impulse, $x[n] = C \delta[n]$, with amplitude $C$ being the initial
value for the iteration. Let's start out with the Fibonacci sequence,
which you'll remember to be:
\begin{equation}
  F[n] = \left\{ \begin{array}{ll}
      1 & n < 2 \\
      F[n-1] + F[n-2] & n \geq 2
    \end{array} \right.
\end{equation}

We can rewrite this recurrence relation as:
\begin{equation}
  y[n] = y[n-1] + y[n-2] + x[n]
  \label{eq:fib}
\end{equation}
and we will get the Fibonacci sequence \emph{if} we input an impulse
(hence, the appearance of the $x[n]$ on the right hand side, which
serves only to initialize the filter). In Matlab, this can be don
trivially by taking a vector of all zeros --- let's call this vector
\verb|x| --- and setting its first value only (\verb|x(1)|) to
$C$. This is also a very good demonstration of the first ``I'' in the
acronym ``IIR'': the impulse response of this filter has infinite
duration.

\paragraph{Step 1.1} If we set $x[n]=\delta[n]$ in~(\ref{eq:fib}), we
should see that the impulse response of this filter is indeed the
Fibonacci sequence. Implement this filter in Matlab and verify that its
impulse response is the Fibonacci sequence. What are the values of the
coefficients that you used?


\paragraph{Step 1.2} What is the value for $n=19$ (\verb|y(20)| in
Matlab)?


\paragraph{Step 1.3} Is this filter stable?


\paragraph{Step 1.4} Let's do something similar with the recurrence
relation for computing the series $y[n] = 1/3^n$ in the text (as
always, remember that Matlab indices start at 1). Set the
coefficients for a feedback filter to implement equation~(5-42) in the
text, $y[n] = 1/3 y[n-1] + x[n]$. What are the filter coefficients?


\paragraph{Step 1.5} What are the pole location(s) for this filter?


\paragraph{Step 1.6} Now use Matlab to calculate the impulse repsonse.
Set the amplitude of the input impulse to be 0.99996. Is this filter
stable?  Is its impulse response consistent with the result of
iterating equation~(5-42) in the notes?


\subsection{Telephone Touch Tone Dialing}
Telephone touch pads generate dual tone multi frequency (DTMF) signals
to dial a telephone. When any key is pressed, the tones of the
corresponding column and row in the table below are generated, hence
it is a ``dual tone'' code. As an example, pressing the 5 button
generates the tones 770Hz and 1336Hz summed together.

\begin{center}
  \begin{tabular}{l|ccc}
    & 1209Hz & 1336Hz & 1477Hz \\ \hline
    697Hz &   1    &   2    &   3    \\
    770Hz &   4    &   5    &   6    \\
    852Hz &   7    &   8    &   9    \\
    941Hz &   $\ast $    &   0    &   \#    
  \end{tabular}
\end{center}

The frequencies in the table above were chosen to avoid harmonics. No
frequency is a multiple of another, the difference between any two
frequencies does not equal any of the frequencies, and the sum of any
two frequencies does not equal any of the frequencies.\footnote{More
  information can be found at:
  \url{http://en.wikipedia.org/wiki/DTMF}} This makes it easier to
detect exactly which tones are present in the dial signal in the
presence of line distortions.

It is possible to decode such a signal by first using a \emph{filter
  bank} composed of seven bandpass filters, one for each of the
frequencies above. When a button is pressed, it will produce a
combination of two tones, and thus, at the decoder end, two of the
bandpass filters will produce significantly higher outputs than the
others. A good measure of the output levels is the average power at
the filter outputs. This is calculated by squaring the filter outputs
and averaging over a short time interval.

\paragraph{Step 2.1} First of all, please write a Matlab
\verb|DTMFCoder| function. This function should take in one argument
--- a telephone key number --- and return a digital waveform
containing the appropriate summed tones. Internally, it should do this
by generating AnalogSignals, summing them, and then sampling them at
8kHz and quantizing them at 16 bits. DTMF signal duration should be
1s. For each of the seven tone frequencies in Hz, what is the
corresponding digital frequency in the range $[0, \pi]$?

\paragraph{Step 2.2} In this step, please construct a bandpass filter
for the 697Hz tone. Use a feedback filter with complex conjugate
poles. Locate these complex conjugate poles at the correct location
for $\pm$697Hz. Use equation~(5-38) of section~5.1.4 of the text to
set the radius of those poles so that the closest other tone
frequency, 770Hz, lies outside the passband (in other words, to set
the bandwidth so that it is significantly smaller than twice the
difference between 697Hz and 770Hz). What were your pole locations?

Compute the corresponding filter coefficients. You can verify your
filter performance by plotting its frequency response.

Verify that the filter output for \verb|DTMFCoder| output for keys 1,
2, and 3 are pretty much identical, and that all other buttons produce
much lower amplitude output.  In your report, include a plot of the
filter output for one of the buttons 1, 2, or 3 and a plot for one of
the buttons 4, 5, or 6.


\paragraph{Step 2.3} Now we are ready to decide whether a particular
frequency is present. Write a Matlab \verb|RMS| function that takes a
vector as input and returns a scalar root mean squared value for it
--- this function should square each value of the vector, take the
mean of those squares, and then take the square root of that
mean. Determine the RMS filter output for telephone buttons 1, 2, and
3 and compare them to the other phone buttons. You should see a much
higher value for 1, 2, and 3 than the other buttons; additionally, the
RMS values for those three buttons should be almost identical. What
are the RMS values you get for pressing 1 versus 4?


\paragraph{Step 2.4} Now we will assemble a filter bank. Implement
filters in Matlab for each of the six other DTMF frequencies and
verify that they work as expected. Now, write a Matlab function
\verb|DTMFDecoder| that takes a single input --- a vector (for which
you will use the \verb|DTMFCoder| output).  \verb|DTMFDecoder| should
compute the RMS value of the output of each of the seven filters in
the filter bank. It should output (to the Matlab console) these RMS
values, and then detect the two highest values. It should use a lookup
table or equivalent logic to decode which button was ``pressed,'' and
output (to the Matlab console) that button.


% LocalWords:  WebQ MATLAB DSP
