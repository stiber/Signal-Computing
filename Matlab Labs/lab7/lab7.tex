%&LaTeX

\section{Joe Fourier Was Not a Discrete Fellow}

\subsection{Lab Background}
By the end of this lab you should have a firm understanding of how the
Discrete Fourier Transform (DFT) can be implemented exactly using the
Fast Fourier Transform (FFT). In addition you should be able to
identify common problems using the DFT to analyze signals. You will
also be familiar with a new tool, the spectrogram, that uses the DFT
as a function of time.

\subsection{Implementing the DFT}

Recall that the DFT can be implemented directly from the analysis
equation. For a length $N$ signal $x[n]$,
\begin{align}
X[k]=\sum_{n=0}^{N-1}x[n]e^{-j \frac{2\pi}{N} nk} && \text{for $k = 0, 1, 2, \cdots N-1$}
\label{eq:dft}
\end{align}
The order of the implementation is $O(N)=N^2$. The following Java code
outlines implementation of a 256-point DFT. It is written without any
algorithmic speedup (i.e., it exactly mirrors equation \ref{eq:dft}).
\begin{lstlisting}[language=Java,basicstyle=\mlttfamily\small]
public class MyDFT
{
  // x is the input and y is the magnitude of the complex DFT
 public void computeDFT(double[] x, double[] y)
  {
  double[] yImag = new double[256];
  double[] yReal = new double[256];

  double twoPiOverN = 2*Math.PI/256;

  for (int k = 0 ; k < 256 ; k++)
  {
    yReal[k] = 0;
    yImag[k] = 0;
    for (int n = 0 ; n < 256 ; n++)
    {
      yReal[k] += x[n]*Math.cos(n*k*twoPiOverN);
      yImag[k] += -x[n]*Math.sin(n*k*twoPiOverN);
    }
    y[k] = Math.sqrt(yReal[k]*yReal[k] + yImag[k]*yImag[k]);
  }
 }
}
\end{lstlisting}

Note that, unlike in Matlab, there is no native support in Java for
complex numbers so this arithmetic is written out explicitly in the
code above. For example, the equation $y=x\times e^{a}$ (where x is a
real number) must be explicitly written out using Euler's formula, and
the real and imaginary portions saved in separate variables,
$y_{real}=x\times \cos(a)$ and $y_{imag}=x\times \sin(a)$.

The FFT algorithm can be used to reduce the computation time of the
DFT to $O(N)=N\log_2 N$ --- a significant speedup for even modest
length signals.

\subsection{The FFT in Matlab}

Matlab includes a \verb|fft| function (there are many more related
operations in the Signal Processing Toolbox, but we are sticking to
``vanilla'' Matlab here). You can look at the documentation for this
function; pay especial attention to the frequency values that
correspond to each element of the vector that this function returns,
and to how to specify the number of points in the FFT it computes.

\paragraph{Step 1.1} Implement your own \verb|myFFT| function in
Matlab that takes a real-valued vector as input, computes a 256-point
FFT, and returns a real-valued vector that is the magnitude of the
(single-sided, i.e., only positive frequencies) FFT. Implement this
function using loops (i.e., do not use recursion). The first part of
this code performs a bit reversal on the input array. You can use the
following code to perform the bit reversal efficiently (efficient bit
reversal algorithms in other languages are typically more complex):
\begin{lstlisting}[style=Matlab-editor,basicstyle=\mlttfamily\small]
% Assume that you want to do a bit reversal of the contents of the vector x
indices = [0 : length(x)-1];                 % binary indices need to start at zero
revIndices = bin2dec(fliplr(dec2bin(indices, 8))); % bit reversed indices
revX = x(revIndices+1);                      % Add 1 to get Matlab indices
\end{lstlisting}
This code converts the vector of in-order indices to an array of
8-character strings (representing those indices as binary 8-bit
numbers). Each string is then reversed, and converted back to
numbers. The resultant vector of indices (which is what they are after
we add one to each) is applied to the signal to pull its entries out
into a new vector, with the order of the entries in bit-reversed
order.

Once you've done this, all you need to do is iterate over the array
$\log_2 N$ times! Remember that you're performing complex arithmetic
and to compute the magnitude of the output array once the FFT is
computed. Include a copy of your Matlab code in your report.

\paragraph{Step 1.2} 
Check your results using the Matlab \verb|fft| function. Your results
should be quite similar, if not identical; this should be apparent by
comparing graphs of the outputs. Take the FFT of a sinusoid with a
frequency of $\pi/4$ radians per second using your FFT implementation
and the \verb|fft| function.



\subsection{Using the DFT}

\paragraph{Step 2.1} 
Create a sum of two sinusoids. Use the \verb|fft| function (it will
allow you, among other things, to apply an $n$-point FFT to a signal
with more than $n$ samples) to compute the FFT of the sum and then
plot the FFT magnitude. Use frequencies of $0.13\pi$ and $0.19\pi$ for
the two sinusoids. Make sure that there are at least 256 samples in
each waveform, as you'll want to use a 256-point FFT! What does the
result look like? Does this make sense?


\paragraph{Step 2.2} At what index (or indices) does the FFT magnitude
reach its peak value(s)? What frequency (or frequencies) does this
correspond to?


\paragraph{Step 2.3} Change the frequencies of the sinusoids to
$0.13\pi$ and $0.14\pi$. Repeat steps 2.1 and 2.2. Do the results
still make sense?


\paragraph{Step 2.4} Replace the sum of two sinusoids with your
\verb|DTMFCoder| function from lab 6. Input a selection of button
values. Does the FFT block show the separate frequencies for each
button?



\subsection{Spectrograms}

Comparing FFT graphs (as in the step 2.4 above) can be difficult to
do. But what if we could analyze the frequency content of a signal as
a function of time? That would make it easier to see differences in
frequency if a signal started changing (like a string of DTMF keys
pressed in turn). To do this we will need a new tool called the
\emph{Spectrogram}. The spectrogram is simply an algorithm for
computing the FFT of a signal at different times and plotting them as
a function of time. The spectrogram is computed in the following way:

\begin{enumerate}
\item A given signal is ``windowed.'' This means that we only take a
  certain number of points from the signal (for this example assume we
  are using a window of length 128 points). To start out, we take the
  first 128 points of the signal (points 1 through 128 of the input
  vector).
\item Take the FFT of the window and save the it in a separate array.
\item Advance the window in time by a certain number of points. For
  instance we can advance the window by 64 points so that we now have
  a window of indices 65 through 192 from the input signal array.
\item Repeat steps 1--3, saving the FFT of each window, until there
  are no longer any points in the input array.
\item Form a 2-D matrix whose columns are the FFT magnitudes of each
  window (placed in chronological order). In this way, each row
  represents a certain frequency, each column represents a given
  instant in time, and the value of the matrix represents the
  magnitude of the FFT.
\end{enumerate}

The result is called a \emph{spectrogram} and is usually displayed as
an RGB image where blue represents small FFT magnitudes and red
represents larger FFT magnitudes (the \emph{Jet} colormap if you are
familiar with color visualizations). There is an art to choosing the
correct parameters of the spectrogram (i.e., window size, FFT size,
how many points to advance the FFT, etc.). Each parameter has
tradeoffs for the time and frequency resolution of the resulting
spectrogram. For our purposes here, we will not be concerned with
these tradeoffs. Instead we will be more interested in getting
familiar with analysis using spectrograms.

The Matlab Signal Processing Toolbox has a \verb|spectrogram|
function, but you can retrieve a drop-in replacement for it from
\url{http://www.ee.columbia.edu/ln/rosa/matlab/sgram/} that will work
without that toolbox. See the online Matlab documentation for the
\verb|spectrogram| function to understand what the parameters to that
function are (though, for our purposes, you should just be able to use
the call \verb|y = myspecgram(x)|.

\paragraph{Step 3.1} Use your \verb|DTMFCoder| function again. Instead
of computing its FFT, compute its spectrogram instead. What does the
spectrogram look like for button 1? Are both frequencies present?


\paragraph{Step 3.2} For this step, use \verb|DTMFCoder| to generate
the codes for multiple buttons --- at least three --- and concatenate
them together to form a single vector. Does the spectrogram make it
easier to judge the frequency content of the keys? Can you clearly see
when the signal changes from one key to another?

% LocalWords:  WebQ MATLAB DSP
